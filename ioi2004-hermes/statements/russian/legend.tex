В современном городе греческих богов улицы расположены в виде целочисленной решетки
и параллельны осям x и y. Для каждого целого числа $Z$ существуют горизонтальная улица
c $y=Z$ и вертикальная улица с $x=Z$. В каждой точке с целочисленными координатами
находится уличный перекресток. Таким образом, каждая пара целых чисел задает
некоторый перекресток. В жаркие дни боги отдыхают в кафетериях, расположенных
на уличных перекрестках. Посыльный Гермес должен послать световые сообщения богам,
отдыхающим в кафетериях, перемещаясь только по улицам города. Каждое сообщение
предназначено только для одного бога, но ничего не случится, если его увидят другие боги.

Сообщения должны быть посланы богам строго в заданном порядке, поэтому Гермесу даны
координаты кафетериев именно в этом порядке. Гермес стартует из точки с координатами $(0, 0)$.
Для того, чтобы послать сообщение в кафетерий с координатами $(X_i, Y_i)$,
Гермесу достаточно посетить некоторую точку на этой же горизонтальной улице 
(с y-координатой $Y_i$) или на этой же вертикальной улице (с x-координатой $X_i$).
После отправки всех сообщений Гермес останавливается.

Вы должны написать программу, которая по заданной последовательности кафетериев находит 
минимальную суммарную длину пути, который должен пройти Гермес для посылки всех сообщений.
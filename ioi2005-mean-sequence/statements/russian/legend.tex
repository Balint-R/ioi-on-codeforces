Рассмотрим неубывающую последовательность $s_1, \ldots, s_{n+1}$ ($s_i \le s_{i + 1}$).
Последовательность $m_1, \ldots, m_n$, в которой каждый член определен как 
$m_i = \frac{s_i + s_{i + 1}}{2}$, назовем {\it средней последовательностью} для последовательности 
$s_1, \ldots, s_{n + 1}$. Например, средняя последовательность для последовательности 
\texttt{1, 2, 2, 4} есть \texttt{1.5, 2, 3}. Заметим, что элементы средней последовательности могут быть дробными числами. 
Тем не менее, в данной задаче используются только те средние последовательности, в которых все числа целые.
Для заданной неубывающей последовательности из n целых чисел $m_1, \ldots, m_n$
необходимо вычислить количество всех неубывающих последовательностей из $n + 1$ целых чисел
$s_1, \ldots, s_{n + 1}$, для которых заданная последовательность $m_1, \ldots, m_n$ является средней последовательностью.

Напишите программу, которая:

\begin{itemize}
\item читает из стандартного ввода неубывающую последовательность целых чисел;
\item вычисляет количество всех неубывающих последовательностей целых чисел, для которых заданная последовательность является средней последовательностью;
\item выводит ответ в стандартный вывод.
\end{itemize}


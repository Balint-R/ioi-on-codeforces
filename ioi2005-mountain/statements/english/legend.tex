The Mountain Amusement Park has opened a brand-new simulated roller coaster. The simulated track consists of $n$ rails attached end-to-end with the beginning of the first rail fixed at elevation $0$. Byteman, the
operator, can reconfigure the track at will by adjusting the elevation change over a number of consecutive
rails. The elevation change over other rails is not affected. Each time rails are adjusted, the following track is
raised or lowered as necessary to connect the track while maintaining the start at elevation $0$. The figure on
the next page illustrates two example track reconfigurations.

Each ride is initiated by launching the car with sufficient energy to reach height $h$. That is, the car will
continue to travel as long as the elevation of the track does not exceed $h$, and as long as the end of the track
is not reached.

Given the record for all the day's rides and track configuration changes, compute for each ride the number
of rails traversed by the car before it stops.

Internally, the simulator represents the track as a sequence of $n$ elevation changes, one for each rail. The
$i$-th number $d_i$ represents the elevation change (in centimetres) over the $i$-th rail. Suppose that after traversing
$i-1$ rails the car has reached an elevation of $h$ centimetres. After traversing $i$ rails the car will have reached
an elevation of $h+d_i$ centimetres.

Initially the rails are horizontal;  that is, $d_i = 0$ for all $i$. Rides and reconfigurations are interleaved through
out the day. Each reconfiguration is specified by three numbers: $a$, $b$ and $D$. The segment to be adjusted
consists of rails a through $b$ (inclusive). The elevation change over each rail in the segment is set to $D$. That
is, $d_i = D$ for all $a \le i \le b$.



Each ride is specified by one number $h$~--- the maximum height that the car can reach.

Write a program that:

\begin{itemize}
\item reads from the standard input a sequence of interleaved reconfigurations and rides,

\item for each ride computes the number of rails traversed by the car,

\item writes the results to the standard output.
\end{itemize}
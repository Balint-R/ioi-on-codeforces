You are given a grader that provides the following functions

\begin{verbatim}
int throwBall(int holeIn, int sideIn, int &holeOut, int &sideOut);
\end{verbatim}

Throws a ball into the box through hole number holeIn in side sideIn, sides are numbered as 1 --- Top, 2 --- Right, 3 --- Bottom and 4 --- Left. Holes are numbered from left to right and from top to bottom starting from number 1 in
each side. In holeOut and sideOut you'll receive the hole and side number where the ball exits the box. The function throwBall returns the number of beeps caused
by the ball hitting a deflector.

\begin{verbatim}
void ResetBox();
\end{verbatim}

Resets every deflector in the box to its initial position.

\begin{verbatim}
void ToggleDeflectors();
\end{verbatim}

Toggles every deflector in the box.

You need to implement function

\begin{verbatim}
void solve(vector<string>& ans);
\end{verbatim}

This function should put answer in to given vector. $ans[i][j]$ should be equal to content of cell $(i, j)$ with same symbols, that are used for input below. If you can't determinate content of cell, you can fill it with \t{?}, and still receive part of points. Size of vector is equal to size of field. Initially all elements are filled with \t{?}

A sample interaction for the box in the previous figure could be:

throwBall(3, 4, holeOut, sideOut);

A ball is thrown into hole number 3 (third from the top) on the left side. Returns 1, indicating that the ball hit 1 deflector. When the function returns, holeOut will equal 2 and sideOut will equal 3 indicating that the ball exited through hole 2 (second from the left) of the bottom side of the box.


Grader will read input in following format:

LINE 1: Contains n, the number of holes on each side.

n LINES: Each line describes a row of the box, starting from the topmost row 
to the bottom row. Each line must contain exactly n characters; each 
character corresponds to a column (running from left to right). 
\begin{itemize}
\item ``\t{.}'' means that the square is empty.
\item ``\t{/}'' means the square contains a deflector with initial position ``/''
\item ``\t{\textbackslash}'' means the square contains a deflector with initial position ``\t{\textbackslash}''
\end{itemize}

Input for test 01 can be found in problem graders archive. Solving it will not be rewarded by any points. 
Вам дан грейдер, предоставляющий следующие функции:

\begin{verbatim}
int throwBall(int holeIn, int sideIn, int &holeOut, int &sideOut);
\end{verbatim}

Вбрасывает шар в ящик через отверстие с номером \t{holeIn} на стороне \t{sideIn}. Стороны пронумерованы следующим образом: \t{1} --- Верхняя, \t{2} --- Правая, \t{3} --- Нижняя и \t{4} --- Левая. Отверстия пронумерованы слева направо и сверху вниз, начиная с номера $1$ на каждой стороне. В параметрах \t{holeOut} и \t{sideOut} функция возвращает номер отверстия и номер стороны --- место, где шар вылетел наружу. В качестве своего значения функция \t{throwBall} возвращает количество звуковых сигналов, вызванных попаданиями шара в отражатели.

\begin{verbatim}
void ResetBox();
\end{verbatim}

Возвращает все отражатели в исходное положение. 

\begin{verbatim}
void ToggleDeflectors();
\end{verbatim}

Меняет положения всех отражателей в ящике на противоположные. 

Вам нужно реализовать функцию:

\begin{verbatim}
void solve(vector<string>& ans);
\end{verbatim}

Эта функция должна записать ответ в данный \t{vector}. \t{ans[i][j]} должны представлять содержимое ячеек $(i, j)$ с помощью символов, указанных ниже. Если вам не удалось определить содержимое ячейки, вы можете заполнить её символом ``\t{?}'', и получить часть символов. Размер \t{ans} совпадает с размером поля. Изначально все элементы равны ``\t{?}''

Пример использования:

\begin{verbatim}
throwBall(3, 4, &holeOut, &sideOut);
\end{verbatim}

Шар вбрасывается в отверстие с номером $3$ (третье сверху) на левой стороне. Возвращается значение $1$ --- это означает, что шар отразился один раз. После возврата из функции, переменная \t{holeOut} содержит значение $2$, а переменная \t{sideOut} содержит значение $3$, то есть шар вылетел через отверстие с номером $2$ (второе слева) нижней стороны ящика. 

Грейдер будет читать данные в следующем формате:

Строка $1$: Число $n$ (размер коробки)

Строки $2\dots n+1$ - каждая строка описывает одну строку чёрного ящика и должна иметь длину $n$. $i$-й символ в строке описывает клетку, находящуюся в соответствующем столбце:

\begin{itemize}
\item <<\t{.}>> означает, что соответствующая клетка пуста.
\item <<\t{/}>> означает, что в клетке находится отражатель с начальным положением <</>>
\item <<\t{\textbackslash}>> означает, что в клетке находится отражатель с начальным положением <<\t{\textbackslash}>>
\end{itemize}


Входные данные для теста 01 можно найти в архиве грейдера. Его решение не будет оценено.
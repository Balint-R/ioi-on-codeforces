СТРОКА 1: Содержит шесть целых чисел, разделенных
пробелами, в следующем порядке: $m$ ($3 \le m \le 1000$), $n$ ($3 \le n \le 1000$), $a$ ($3 \le a \le m$),
$b$ ($3 \le b \le n$), $c$ ($1 \le c \le a - 2$) и $d$ ($1 \le d \le b-2$).


СЛЕДУЮЩИЕ $n$ СТРОК: Каждая из этих строк файла содержит $m$
целых чисел, разделенных пробелами. Эти числа
соответствуют высотам клеток в одной строке сетки. Первая
из этих строк соответствует верхней строке (строке 1) сетки,
а последняя~--- нижней строке (строке $n$). При этом $m$ чисел в
каждой строке соответствуют высотам клеток этой строки,
начиная со столбца 1. Все высоты~--- целые числа от 1 до 100. 
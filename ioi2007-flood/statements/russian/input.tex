Первая строка входных данных содержит целое число  $N$ ($2 \le N \le 100\,000$), количество точек на 
плоскости. 

Каждая из следующих $N$ строк содержит по два целых числа $X$ и $Y$ (каждое от $0$ до $1\,000\,000$
включительно) "--- координаты точки. Точки нумеруются от $1$ до $N$ в том порядке, в котором они заданы. 
Никакие две точки не совпадают. 

Следующая строка содержит целое число $W$ ($1 \le W \le 2 \cdot N$), количество стен. 

Каждая из следующих $W$ строк содержит по два различных целых числа $A$ и $B$ ($1 \le A, B \le N$), 
означающие, что перед наводнением существовала стена, соединяющая точки $A$ и $B$. Стены нумеруются 
от $1$ до $W$ в том порядке, в котором они заданы. 

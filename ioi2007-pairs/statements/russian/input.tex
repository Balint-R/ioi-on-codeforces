Первая строка входных данных содержит четыре целых числа в следующем порядке: 

\begin{itemize}
\item Тип доски $B$ $(1 \le B \le 3)$;

\item Количество зверюшек $N$ $(1 \le N \le 100\,000)$; 

\item Максимальное расстояние $D$, на котором две зверюшки могут слышать друг друга
$(1 \le D \le 100\,000\,000)$;

\item Размер доски $M$ (максимальная координата, которая может встретиться во входных данных):

\begin{itemize}
\item Если $B=1$, то $M$ не будет превышать 75\,000\,000. 
\item Если $B=2$, то $M$ не будет превышать 75\,000. 
\item Если $B=3$, то $M$ не будет превышать 75. 
\end{itemize}
\end{itemize}

Каждая из следующих $N$ строк содержит по $B$ целых чисел, разделенных одиночными пробелами~---
координаты соответствующей зверюшки. Каждая координата находится в диапазоне от $1$ до $M$,
включительно. 

В одной ячейке может располагаться более одной зверюшки.

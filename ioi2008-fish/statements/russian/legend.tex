Как рассказывала Шахерезада, в одной далёкой стране, посреди пустыни, есть
озеро. Вначале в том озере было $F$ рыбок. Были выбраны $K$ ценнейших видов
драгоценных камней, и каждая рыбка проглотила один камень. Заметьте, что
$K$ может быть меньше $F$, поэтому некоторым рыбкам достались камни одного
вида.

Время шло, рыбки охотились друг на друга. Одна рыбка может съесть другую,
если та хотя бы в два раза короче (рыбка $A$ может съесть рыбку $B$ тогда
и только тогда, когда $L_A \ge 2 \cdot L_B$). Рыбки выбирают своих жертв без
каких-то особых правил. Одна рыбка может выбрать несколько меньших рыбок
и съесть их одну за одной, а может и не съесть ни одной, даже если способна.
Когда рыбка съедает другую, её длина не меняется, однако все камни из живота
меньшей рыбки оказываются в животе большей.

Шахерезада рассказывала, что тот, кто сумеет отыскать озеро, получит право
поймать одну рыбку оттуда и забрать все камни, оказавшиеся у неё в животе.
Вы решили испытать удачу, но перед тем, как отправиться в путь,
заинтересовались тем, сколько различных комбинаций камней можно получить,
поймав одну рыбку.

Известно, камень какого вида проглотила каждая рыбка вначале. Вычислите
количество комбинаций камней, которые можно получить, поймав ровно одну рыбку,
по модулю $M$. Комбинация однозначно определяется количеством камней каждого
вида. Любые два камня одного вида неразличимы, порядок камней не имеет
значения.

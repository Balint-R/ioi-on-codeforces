Всего $11$ возможных комбинаций, поэтому нужно вывести $11$ по модулю $7$, то
есть $4$.

Возможные комбинации: $[1]$ $[1,2]$ $[1,2,3]$ $[1,2,3,3]$ $[1,3]$ $[1,3,3]$
$[2]$ $[2,3]$ $[2,3,3]$ $[3]$ и $[3,3]$. (Для каждой комбинации перечислены
камни, её составляющие. К примеру, $[2,3,3]$ "--- комбинация, состоящая из
одного камня вида $2$ и двух камней вида $3$.)

Эти комбинации можно получить следующими способами:

\begin{shortitems}
  \item $[1]$: Можно поймать вторую (или четвёртую) рыбку, пока она не съела
    какую-то ещё.
  \item $[1,2]$: Если вторая рыбка съест первую рыбку, в ней будут камень вида
    $1$ (который она изначально проглотила), а также камень вида $2$ (из первой
    рыбки).
  \item $[1,2,3]$: Один из возможных способов получения этой комбинации:
    четвёртая рыбка съедает первую, потом третья съедает четвёртую. Если теперь
    поймать третью рыбку, у неё в животе найдётся по одному камню всех видов.
  \item $[1,2,3,3]$: Четвёртая ест первую, третья ест четвёртую, третья ест
    пятую, вы ловите третью.
  \item $[1,3]$: Третья ест четвёртую, вы её ловите.
  \item $[1,3,3]$: Третья ест пятую, третья ест четвёртую, вы её ловите.
  \item $[2]$: Вы ловите первую рыбку.
  \item $[2,3]$: Третья ест первую, вы её ловите.
  \item $[2,3,3]$: Третья ест первую, третья ест пятую, вы её ловите.
  \item $[3]$: Вы ловите третью рыбку.
  \item $[3,3]$: Третья ест пятую, вы её ловите.
\end{shortitems}

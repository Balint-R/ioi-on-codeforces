Медведь Миша нашел маленькое сокровище~--- спрятанный горшочек меда! Он c удовольствием
поедал мед, но вдруг одна пчела его заметила и забила тревогу. Миша знает, что после этого пчелы вылетят из своих ульев и будут разлетаться вокруг, чтобы настичь его. Он знает, что нужно бросать горшочек и быстро идти домой, но мед так сладок, что Миша не хочет бросать его есть слишком рано. Помогите Мише определить последний возможный момент, когда он может прекратить есть мед.

Лес представлен картой в виде квадратной сетки, которая состоит из $N \times N$ единичных ячеек, стороны которых параллельны направлениям <<север-юг>> и <<запад-восток>>. Каждая ячейка леса занята либо деревом, либо травой, либо ульем, либо Мишиным домом. Две ячейки называются смежными, если одна из них находится непосредственно к северу, югу, востоку или западу от другой, но не по диагонали. Миша неповоротлив, поэтому он может перемещаться только в смежную ячейку. Миша может перемещаться только по ячейкам с травой и не может перемещаться по ячейкам с деревьями или ульями. Также он не может перемещаться больше, чем на $S$ ячеек в минуту.

В момент, когда прозвучала тревога, Миша находится в ячейке с травой, где он нашел горшочек с медом, а все пчелы~--- в ячейках, где расположены ульи (в лесу может быть больше одного улья). С этого момента, на протяжении каждой следующей минуты происходят следующие события в таком порядке: 

\begin{itemize}
\item Если Миша все еще ест мед, он решает, будет ли он продолжать есть или будет уходить. Если он продолжает есть мед~--- он не перемещается всю минуту. Иначе, он немедленно уходит и перемещается по лесу не более чем на $S$ ячеек, как описано выше. Миша не может брать с собой мед, и как только он ушел, он уже не может его есть.
\item Как только Миша заканчивает есть или перемещаться в течение минуты, пчелы разлетаются на одну ячейку дальше, занимая только ячейки с травой. Точнее, пчелы разлетаются во все ячейки с травой, смежные с любой ячейкой, где уже есть пчелы. Как только в ячейке появляются пчелы, они там остаются навсегда (пчёлы не перемещаются, а
распространяются).
\end{itemize}

Другими словами, пчелы разлетаются так: когда звучит тревога, пчелы находятся в ячейках, где расположены ульи. В конце первой минуты они занимают все ячейки с травой, смежные с ульями, и остаются в тех ячейках, где расположены ульи. В конце второй минуты пчелы дополнительно занимают все ячейки с травой, смежные со смежными с ульями ячейками, и так далее. Имея достаточно времени, пчелы займут все ячейки с травой, которые они могут достичь. 

Ни Миша, ни пчелы не могут покидать пределы леса. Также обратите внимание, что согласно описанным правилам, Миша ест мед целое число минут.

Пчелы настигают Мишу, если в какой-то момент времени Миша оказывается в ячейке, занятой
пчелами.

Напишите программу, которая по карте леса определяет наибольшее количество минут, на
протяжении которых Миша может продолжать есть мед в своем исходном расположении, все еще имея возможность попасть домой до того, как пчелы его настигнут.

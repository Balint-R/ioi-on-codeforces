The local Plovdiv Olympiad in Informatics (POI) was held according to the following unusual rules. There were $N$ contestants and $T$ tasks. Each task was graded with only one test case, therefore for every task and every contestant there were only two possibilities: either the contestant solved the task, or the contestant did not solve the task. There was no partial scoring on any task.

The number of points assigned to each task was determined after the contest and was equal to the number of contestants that did not solve the task. The score of each contestant was equal to the sum of points assigned to the tasks solved by that contestant.

Philip participated in the contest, but he is confused by the complicated scoring rules, and now he is staring at the results, unable to determine his place in the final standings. Help Philip by writing a program that calculates his score and his ranking.

Before the contest, the contestants were assigned unique IDs from $1$ to $N$ inclusive. Philip's ID was $P$. The final standings of the competition list the contestants in descending order of their scores. In case of a tie, among the tied contestants, those who have solved more tasks will be listed ahead of those who have solved fewer tasks. In case of a tie by this criterion as well, the contestants with equal results will be listed in ascending order of their IDs.

Write a program that, given which problems were solved by which contestant, determines
Philip's score and his rank in the final standings.
Агентство по Развитию Регионов Организации Объединённых Наций (АРРООН) имеет хорошую
организационную структуру. В АРРООН работают $N$ человек, каждый из которых приехал из одного из $R$ различных географических регионов мира. Регионы пронумерованы от $1$ до $R$ включительно в произвольном порядке. Работники имеют номера от $1$ до $N$ включительно в порядке убывания их возрастов. При этом работник с номером 1 (председатель) является самым старшим из них. Каждый работник, кроме председателя, имеет одного непосредственного начальника. Начальник всегда старше, чем работник, которым он руководит.

Будем говорить, что работник $A$ является менеджером работника $B$, если работник $A$ является либо непосредственным начальником работника $B$, либо менеджером непосредственного начальника работника $B$. Например, председатель является менеджером любого другого работника. Кроме того, очевидно, что никакие два работника не могут быть менеджерами друг друга.

К сожалению, Бюро Исследований Организации Объединённых Наций (БИООН) недавно получило
некоторое количество жалоб, что АРРООН имеет несбалансированную структуру, более выгодную для одних регионов и менее выгодную для других. Чтобы расследовать эти случаи, БИООН собирается создать компьютерную систему, которой будет задаваться описание структуры АРРООН и которая будет отвечать на запросы в следующей форме: по двум регионам $r_1$ и $r_2$ найти количество таких пар работников ($e_1$, $e_2$), что работник $e_1$ приехал из региона $r_1$, работник $e_2$ приехал из региона $r_2$, и $e_1$ является менеджером $e_2$. У каждого запроса два параметра: номера регионов $r_1$ и $r_2$. Результат выполнения запроса~--- одно целое число: количество различных пар ($e_1$, $e_2$), которые удовлетворяют вышеописанным условиям.

Напишите программу, которая по заданным регионам, из которых приехали работники агентства, а также информации о том, кто является чьим непосредственным начальником, интерактивно отвечает на вопросы в вышеописанной форме.
Jack and Jill play a game called Hotter, Colder. Jill has a number between $1$ and $N$,
and Jack makes repeated attempts to guess it.

Each of Jack's guesses is a number between $1$ and $N$. In response to each guess, Jill
answers hotter, colder or same. For Jack's first guess, Jill answers same. For the
remaining guesses Jill answers:

\begin{itemize}
\item hotter if this guess is closer to Jill's number than his previous guess
\item colder if this guess is farther from Jill's number than his previous guess
\item same if this guess is neither closer to nor further from Jill's number than his
previous guess.
\end{itemize}

You are to implement a procedure \t{HC(N)} that plays Jack's role. This implementation
may repeatedly call \t{Guess(G)}, with $G$ a number between $1$ and $N$. \t{Guess(G)} will return $1$ to indicate hotter, $-1$ to indicate colder or $0$ to indicate same. \t{HC(N)} must return Jill's number.

As example, assume $N=5$, and Jill has chosen the number $2$. When
procedure \t{HC} makes the following sequence of calls to Guess, the results in the
second column will be returned.

\begin{tabular}{|l|l|l|} \hline
Call & Returned value & Explanation \\ \hline
\t{Guess(5)} & 0 & Same (first call) \\ \hline
\t{Guess(3)} & 1 & Hotter \\ \hline
\t{Guess(4)} & -1 & Colder \\ \hline
\t{Guess(1)} & 1 & Hotter \\ \hline
\t{Guess(3)} & 0 & Same \\ \hline
\end{tabular}

At this point Jack knows the answer, and \t{HC} should return $2$. It has taken Jack $5$
guesses to determine Jill's number. You can do better.
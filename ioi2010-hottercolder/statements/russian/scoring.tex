{\bf Подзадача 1 [25 баллов]}

Процедура \t{HC(N)} должна вызвать процедуру \t{Guess(G)} не более $500$ раз. Всего будет не более $125\,250$ вызовов процедуры \t{HC(N)}, при $N$ от $1$ до $500$.

{\bf Подзадача 2 [25 баллов]}

Процедура \t{HC(N)} должна вызвать процедуру \t{Guess(G)} не более $18$ раз. Всего будет не более $125\,250$ вызовов процедуры \t{HC(N)}, при $N$ от $1$ до $500$.

{\bf Подзадача 3 [25 баллов]}

Процедура \t{HC(N)} должна вызвать процедуру \t{Guess(G)} не более $16$ раз. Всего будет не более $125\,250$ вызовов процедуры \t{HC(N)}, при $N$ от $1$ до $500$.

HC(N) must call \t{Guess(G)} at most $16$ times. There will be at most $125\,250$ calls to \t{HC(N)}, with N between $1$ and $500$.

{\bf Подзадача 4 [до 25 баллов]}

Обозначим за $W$ максимальное целое число такое, что $2^W \le 3 N$. По этой подзадаче ваше решение наберёт:

\begin{itemize}
\item $0$ баллов, если процедура \t{HC(N)} вызовет процедуру \t{Guess(G)} $2W$ или более раз, 
\item $25\alpha$ баллов, где $\alpha$ --- максимальное вещественное число такое, что $0 < \alpha < 1$ и процедура \t{HC(N)} вызовет процедуру \t{Guess(G)} не более $2W - \alpha W$ раз, 
\item $25$ баллов, если процедура \t{HC(N)} вызовет процедуру \t{Guess(G)} не более $W$ раз
\end{itemize}

Всего будет не более $1\,000\,000$ вызовов процедуры \t{HC(N)}, при $N$ от $1$ до $500\,000\,000$

В южном Онтарио многие фермеры, выращивающие кукурузу,
создают на поле лабиринты из стеблей кукурузы. Лабиринты создаются осенью, когда урожай початков собран. У вас еще есть время помочь разработать лучший лабиринт для 2010 года.

Поле засеяно кукурузой за исключением нескольких препятствий (деревьев, строений и тому подобного), где кукуруза не растет. Стебли, высота которых огромна, образуют стенки лабиринта. Тропинки прокладываются на прямоугольной сетке в ячейках площадью в 1 квадратный метр каждая путём вырубания в них стеблей. Одна такая ячейка сетки на границе является входом в лабиринт. Ещё одна ячейка в лабиринте является целью лабиринта.

Джек посещает кукурузные лабиринты каждый год, поэтому он стал экспертом в быстром нахождении пути от входа к цели. Вы разрабатываете новый лабиринт, и ваша задача заключается в том, чтобы определить, в каких ячейках нужно вырубить стебли, чтобы максимизировать количество ячеек в лабиринте, которые должен пройти Джек.

Система оценивания определит, какая ячейка является входом (единственная из ячеек на границе) и какая ячейка является целью~--- та, до которой Джек должен идти как можно дольше.

Карта прямоугольного поля представлена в виде текста, например, поле размером 6×10 метров, содержащее восемь деревьев, может быть представлено так:

\begin{verbatim}
##X#######
###X######
####X##X##
##########
##XXXX####
##########
\end{verbatim}

Символ \t{\#} задает ячейку со стоящими стеблями кукурузы, а символ \t{X} задаёт ячейку с препятствием (например, деревом), которое нельзя удалить, чтобы проложить тропу.

Поле преобразуется в лабиринт вырубанием квадратных ячеек, засеянных кукурузой. Одна вырубленная ячейка (вход) должна размещаться на краю поля. Остальные вырубленные ячейки должны быть внутри поля. Ваша задача~--- максимизировать кратчайший путь от входа до цели, измеряемый количеством вырубленных ячеек, которые должен пройти Джек, включая ячейки входа и цели. Разрешается переходить из одной ячейки поля в другую только тогда, когда обе эти ячейки вырублены и имеют общую сторону.

В посылаемом на проверку файле вырубленные ячейки должны быть обозначены
символами \t{.} (точки). Ровно одна вырубленная ячейка должна быть на границе поля.

Например:
\begin{verbatim}
#.X#######
#.#X#...##
#...X#.X.#
#.#......#
#.XXXX##.#
##########
\end{verbatim}

Ниже, только в целях удобства иллюстрации, мы отметили вход в лабиринт символом \t{E},
цель лабиринта~--- символом \t{C}, а остальной путь~--- символами \t{+}. Длина пути в этом
случае равна 12 ячейкам.
\begin{verbatim}
#EX#######
#+#X#C+.##
#+++X#+X.#
#.#++++..#
#.XXXX##.#
##########
\end{verbatim}

Это задача с открытыми тестами и частичной системой оценивания каждого теста. Вам дано $10$ входных файлов, описывающих поле. Для каждого входного файла вы должны отправить выходной файл, описывающий полученный лабиринт.

В данной задаче не требуется отправлять какой-либо исходный код на проверку.
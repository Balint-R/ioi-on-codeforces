\begin{center}
\renewcommand{\arraystretch}{1.5}
\begin{tabular}{|c|c|c|c|}
\hline
Подзадача & Баллы & $N$ & \parbox{10cm}{\centering \vspace{2mm}Дополнительные ограничения на входные данные\\\vspace{2mm}}\\
\hline
1 & 25 & $1 \le N \le 1\,000$ & \parbox{9cm}{\centering \vspace{2mm}Предположим, что все города лежат на прямой от востока к западу, и все дороги идут по прямой без ветвлений. Более формально, предположим, что для всех $i$ из диапазона $0 \le i \le N-2$ выполняются соотношения $S[i] = i$ и $D[i] = i+1$.\\\vspace{2mm}}\\
\hline
2 & 25 & $1 \le N \le 1\,000\,000$ & \parbox{9cm}{\centering \vspace{2mm} Предположим, что все города лежат на прямой от востока к западу, и все дороги идут по прямой без ветвлений. Более формально, предположим, что для всех $i$ из диапазона $0 \le i \le N-2$ выполняются соотношения $S[i] = i$ и $D[i] = i+1$.\\\vspace{2mm}} \\
\hline
3 & 25 & $1 \le N \le 1\,000$ & --- \\
\hline
4 & 25 & $1 \le N \le 1\,000\,000$ & --- \\
\hline

\end{tabular}
\end{center}
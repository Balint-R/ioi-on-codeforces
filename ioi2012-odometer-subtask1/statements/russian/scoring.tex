\textbf{Подзадача 1 [9 баллов]}

В начале в клетке $(0, 0)$ находится $x$ камней, и $y$ камней в клетке $(0, 1)$, в то время, как все
остальные клетки пустые. Обратите внимание, что в любой клетке может быть не более $15$
камней. Требуется написать программу, после выполнения которой одометр будет
находится в клетке $(0, 0)$, если $x \le y$, и в клетке $(0, 1)$~--- в противном случае. Нас не
интересует направление ориентации одометра; нас так же не волнует количество камней
и их расположение.

Ограничения: размер программы $\le 100$, количество операций $\le 1\,000$.


\textbf{Подзадача 2 [12 баллов]}

Такая же задача, как вышеописанная, но после завершении программы клетка $(0, 0)$
должна содержать ровно $x$ камней и клетка $(0, 1)$ должен содержать ровно $y$ камней.

Ограничения: размер программы $\le 200$, количество операций $\le 2\,000$.



\textbf{Подзадача 3 [19 баллов]}

В строке с индексом $0$ находится ровно два камня. Один~--- в клетке $(0, x)$, другой~--- в клетке
$(0,y)$. $x$ и $y$ различны, $(x + y)$~--- четно. Требуется написать программу, в результате
выполнения которой одометр окажется в клетке ($0, (x + y) / 2)$, то есть, ровно по середине
между двумя клетками, которые содержали камни. Конечное расположение камней не
имеет значения.

Ограничения: размер программы $\le 100$, количество операций $\le 200\,000$.


\textbf{Подзадача 4 [до 32 баллов]}

На доске находится не более $15$ камней, никакие две из них не находятся в одной клетке.
Требуется написать программу, которая собирает все камни в северо-западном углу. Если
быть более точным, то, если исходно на доске было $x$ камней, тогда после завершения
программы в клетке $(0, 0)$ должно находиться ровно $x$ камней, а в остальных клетках
камней не должно быть.

Количество баллов за эту подзадачу зависит от количества операций, совершенных
посланной на проверку программой. В частности, если $L$~--- это максимальное количество
операций для различных входных данных, соответствующих этой подгруппе, то вы
получите следующее количество баллов:
\begin{itemize}
\item $32$ балла, если $L \le 200 000$;
\item $32 - 32 \log_{10} (L / 200 000)$ баллов, если $200\,000 < L < 2\,000\,000$;
\item $0$ баллов, если $L \ge 2\,000\,000$.
\end{itemize}

Ограничения: размер программы $\le 200$.


\textbf{Подзадача 5 [до 28 баллов]}

В каждой клетке может быть любое количество камней (разумеется, от $0$ до $15$). Требуется
написать программу, которая находит минимум, то есть после её завершения одометр
находится в такой клетке $(i, j)$, что любая другая клетка содержит как минимум столько же
камней, сколько и клетка $(i, j)$. После выполнения программы количество камней в
каждой клетке должно быть таким же, как и до запуска программы.

Количество баллов за эту подзадачу зависит от размера программы $P$. В частности, вы
получите следующее количество баллов:
\begin{itemize}
\item $28$ баллов, если $P \le 444$;
\item $28 - 28 log_{10} (P / 444)$ баллов, если $444 < P < 4\,440$;
\item $0$ баллов, если $P \ge 4\,440$.
\end{itemize}

Ограничения: количество операций $\le 44\,400\,000$.
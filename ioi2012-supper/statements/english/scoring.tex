\begin{tabular}{|c|c|lll|}
\hline
\bf{Subtask}&\bf{Points}&\multicolumn{3}{c|}{\bf{Conditions}}\\\hline
1 & 8 & $N \le 5\,000$ & $M = 65\,000\,bits$&\\\hline
2 & 9 & $N \le 100\,000$ & $M = 2\,000\,000\,bits$&\\\hline
3 & 9 & $N \le 100\,000$ & $M = 1\,500\,bits$ & $K \le 25\,000$\\\hline
4 & 35 & $N \le 5\,000$ & $M = 10\,000\,bits$&\\\hline
5 & up to 39 & $N \le 100\,000$ & $M = 1\,800\,000\,bits$ & $K \le 25\,000$\\\hline
\end{tabular}

The score for the last subtask depends on the length $R$ of the advice your program communicates. More precisely, if $R_{max}$ is the maximum (over all test cases) of the length of the advice sequence produced by your routine \t{ComputeAdvice}, your score will be:
\begin{itemize}
\item $39$ points if $R_{max} \le 200\,000$;
\item $\frac{39*(1\,800\,000 - R_{max})}{1\,600\,000}$ points if $200\,000 < R_{max} < 1\,800\,000$;
\item $0$ points if $R_{max} \geq 1\,800\,000$.
\end{itemize}
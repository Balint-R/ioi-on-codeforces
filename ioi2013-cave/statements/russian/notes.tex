Предположим, что двери и переключатели расположены, как показано на рисунке выше. Возможный порядок вызова функций представлен ниже.
\begin{center}
\renewcommand{\arraystretch}{2.2}
\begin{tabular}{ |c|c|c| }
\hline
Вызов функции & \parbox{3cm}{\centering \vspace{2mm}Возвращаемое значение\\\vspace{2mm}} & Пояснение\\
\hline
\t{tryCombination([1, 0, 1, 1])} & $1$ & \parbox{7cm}{\centering \vspace{2mm}Ситуация соответствует показанной на рисунке. Переключатели $0$, $2$ и $3$ находятся в положении <<вниз>>, а переключатель $1$~--- в положении <<вверх>>. Функция возвращает значение $1$, что значит, что дверь $1$~--- первая дверь слева, которая закрыта.\\\vspace{2mm}}\\
\hline
\t{tryCombination([0, 1, 1, 0])} & 3 & \parbox{7cm}{\centering \vspace{2mm}Двери $0$, $1$ и $2$ открыты, а дверь $3$~--- закрыта.\\\vspace{2mm}}\\
\hline
\t{tryCombination([1, ­1, 1, 0])} & $-1$ & \parbox{7cm}{\centering \vspace{2mm}Перевод переключателя $0$ в положение <<вниз>> приводит к тому, что все двери открылись, что обозначается возвращаемым значением ­$-1$.\\\vspace{2mm}}\\
\hline
\t{answer([1, 1, 1, 0],[3, 1, 0, 2])} & \parbox{3cm}{\centering \vspace{2mm}(Программа завершается)\\\vspace{2mm}} & \parbox{7cm}{\centering \vspace{2mm}Ваша программа определила, что комбинация $[1,1, 1,0]$ задаёт правильное положение каждого из переключателей, и переключатели $0$, $1$, $2$ и $3$ соединены с дверями $3$, $1$, $0$ и $2$, соответственно.\\\vspace{2mm}}\\
\hline
\end{tabular}
\end{center}
Ваша программа должна содержать \t{\#include "cave.h"}.
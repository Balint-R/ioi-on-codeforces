Эта история происходила много­много лет назад в те времена, когда о проведении
международной олимпиады по информатике никто даже и не мечтал.

В одной стране живёт Змей. В этой стране есть $N$ озёр, пронумерованных от $0$ до
$N -­ 1$. Кроме того, изначально есть $M$ двусторонних тропинок, соединяющих пары
озёр. По тропинкам может перемещаться Змей. Каждая пара озёр соединена
(напрямую или через промежуточные озёра) не более чем одной последовательностью
изначально заданных тропинок, хотя какие­то пары озёр могут быть вообще не
соединены (таким образом, $M \leq N - ­1$). Для каждой изначально заданной тропинки
известно количество дней, которое тратит Змей, перемещаясь по этой тропинке. Эти
числа могут различаться для разных тропинок.

Друг Змея, Кенгуру, хочет проложить $N -­ M ­- 1$ новую тропинку так, чтобы Змей
смог путешествовать между любой парой озёр. Кенгуру может создавать новую
тропинку между любой парой озёр, и перемещение по каждой новой тропинке,
которую создаёт Кенгуру, занимает у Змея $L$ дней.

Кенгуру хочет сделать путешествия Змея как можно более быстрыми. С этой целью
Кенгуру собирается проложить новые тропинки таким образом, чтобы максимальное
время путешествия между двумя озёрами было как можно меньше. Помогите Кенгуру
и Змею определить максимальное время путешествия между двумя озёрами после
того, как Кенгуру проложит новые тропинки указанным способом.

Ваше решение должно реализовывать нижеописанную функцию $travelTime()$:

\t{int travelTime(int N, int M, int L, int A[], int B[], int T[]);}

Эта функция должна вычислять максимальное время (в днях), которое может
потребоваться для путешествия от одного озера до другого, в предположении, что
Кенгуру добавил $(N -­ M -­ 1)$ новую тропинку таким образом, что все озёра связаны
между собой, и это максимальное время является минимально возможным.

Параметры:
\begin{itemize}
\item $N$: количество озёр.
\item $M$: количество изначально заданных тропинок.
\item $L$: время в днях, которое потребуется Змею на перемещение по любой из проложенных новых тропинок.
\item $A$, $B$ и $T$: массивы длины $M$, которые задают концы и время перемещения по каждой из изначально заданных тропинок, таким образом, что $i$-­я изначально заданная тропинка соединяет озёра $A[i­-1]$ и $B[i - 1]$, а $T[i-­1]$ дней уходит на путешествие по ней в любом направлении.
\item \textit{Возвращаемое значение}: максимальное время путешествия между любой парой озёр, как описано выше.
\end{itemize}

Consider the first example.
Минимальное время, за которое можно убрать все игрушки, --- три минуты. Один из оптимальных способов распределения работы между роботами указан в таблице ниже:
\begin{center}
\begin{tabular}{ |c|c|c|c|c|c| }
\hline
 & \parbox{2cm}{\centering Слабый робот 0} & \parbox{2cm}{\centering \vspace{2mm}Слабый робот 1\\\vspace{2mm}} & \parbox{2cm}{\centering \vspace{2mm}Слабый робот 2\\\vspace{2mm}} & \parbox{2cm}{\centering \vspace{2mm}Маленький робот 0\\\vspace{2mm}} & \parbox{2cm}{\centering \vspace{2mm}Маленький робот 1\\\vspace{2mm}}\\
\hline
Первая минута & Игрушка 0 & Игрушка 4 & Игрушка 1 & Игрушка 6 & Игрушка 2\\
\hline
Вторая минута & Игрушка 5 & & Игрушка 3 & & Игрушка 8\\
\hline
Третья минута & & & Игрушка 7 & & Игрушка 9\\
\hline
\end{tabular}
\end{center}
Во втором примере ни один из роботов не может убрать игрушку весом $5$ и размером $3$, поэтому не существует способа убрать все игрушки.

Ваша программа должна содержать \t{\#include "robots.h"}.
Создается социальная сеть, состоящая из $n$ участников, пронумерованных $0, \ldots, n - 1$. Некоторые пары участников этой сети могут стать друзьями. Если участник $x$ становится другом участника $y$, то участник $y$ также становится другом участника $x$.

Участники добавляются в сеть за $n$ этапов, которые также пронумерованы от $0$ до $n - 1$. Участник $i$ добавляется на этапе $i$. На этапе $0$ добавляется участник с номером $0$ как единственный участник сети. На каждом из следующих $n - 1$ этапов очередной участник добавляется в сеть \texttt{хозяином} этапа, которым может быть любой участник, уже добавленный в сеть. На этапе $i (0 < i < n)$, хозяин этапа может добавить очередного участника $i$ в сеть по одному из трех протоколов:

\begin{itemize}
\item \texttt{IAmYourFriend} делает участника $i$ другом только хозяина этапа.
\item \texttt{MyFriendsAreYourFriends} делает участника $i$ другом \texttt{каждого} друга хозяина в этот момент. Заметьте, что этот протокол \texttt{не} делает участника $i$ другом хозяина.
\item \texttt{WeAreYourFriends} делает участника $i$ другом хозяина в этот момент, а также другом \texttt{каждого} друга хозяина.
\end{itemize}

После того, как сеть создана, необходимо сделать \texttt{выборку} для опроса, то есть отобрать группу участников сети. Поскольку друзья обычно имеют общие интересы, эта выборка не должна содержать пары участников, являющихся друзьями. Каждый участник имеет
некоторый \texttt{уровень доверия} в опросах, который задан положительным целым числом, и нужно сделать выборку участников с максимальным суммарным уровнем доверия.

\textbf{Постановка задачи}

Имея описание каждого этапа и уровень доверия в опросах каждого участника, необходимо
найти выборку участников сети с максимальным суммарным уровнем доверия. Вы должны
реализовать функцию \texttt{findSample}.

\begin{itemize}
\item \texttt{int findSample(int n, int confidence[], int host[], int protocol[])}
\begin{itemize}
\item $n$: количество участников.
\item $confidence$: массив длины $n$; $confidence[i]$ задает уровень доверия к участнику с номером $i$.
\item $host$: массив длины $n$; $host[i]$ задает хозяина $i$-го этапа.
\item $protocol$: массив длины $n$; $protocol[i]$ задает код протокола, используемого на
$i$-м этапе $(0 < i < n)$: $0$~--- для \texttt{IAmYourFriend}, $1$~--- для \texttt{MyFriendsAreYourFriends}, $2$~--- для \texttt{WeAreYourFriends}.
\item поскольку на этапе $0$ нет хозяина, и $host[0]$ и $protocol[0]$ не определены, то
ваша программа не должна к ним обращаться.
\item функция должна возвращать максимально возможный суммарный уровень доверия
для выборки участников.
\end{itemize}
\end{itemize}
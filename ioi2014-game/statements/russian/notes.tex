Мы рассмотрим правила игры на следующих трех примерах. В каждом из них $n = 4$, $r = 6$.

В первом примере (таблица ниже) Джан-Джи проигрывает, потому что после 4-го вопроса,
Мей-Ю уверена, что между любыми двумя городами можно долететь, используя только
авиарейсы, не зависимо от того, какими будут ответы на вопросы 5 и 6.

\begin{center}
\renewcommand{\arraystretch}{1.5}
\begin{tabular}{|c|c|c|}
\hline
Номер & Вопрос & Ответ \\
\hline
1 &  0, 1 & да\\
\hline
2 &  3, 0 & да\\
\hline
3 &  1, 2 & нет\\
\hline
4 &  0, 2 & да\\
\hline
--- &  --- & ---\\
\hline
5 &  3, 1 & нет\\
\hline
6 &  2, 3 & нет\\
\hline
\end{tabular}
\end{center}


В следующем примере Мей-Ю после третьего вопроса может определить, что, как бы ДжанДжи не отвечал дальше на остальные вопросы, нельзя добраться из города 0 в город 1,
используя только авиарейсы. В этом случае Джан-Джи проигрывает снова.

\begin{center}
\renewcommand{\arraystretch}{1.5}
\begin{tabular}{|c|c|c|}
\hline
Номер & Вопрос & Ответ \\
\hline
1 &  0, 3 & нет\\
\hline
2 &  2, 0 & нет\\
\hline
3 &  0, 1 & нет\\
\hline
--- &  --- & ---\\
\hline
4 &  1, 2 & да\\
\hline
5 &  1, 3 & да\\
\hline
6 &  2, 3 & да\\
\hline
\end{tabular}
\end{center}

В последнем примере Мей-Ю не может определить ответ на вопрос до тех пор, пока не
получит ответы на все 6 вопросов, поэтому Джан-Джи выигрывает. В частности, так как
Джан-Джи ответил <<да>> на последний вопрос, то долететь из любого города в любой другой
возможно. Если бы он ответил <<нет>>, то долететь нельзя.

\begin{center}
\renewcommand{\arraystretch}{1.5}
\begin{tabular}{|c|c|c|}
\hline
Номер & Вопрос & Ответ \\
\hline
1 &  0, 3 & нет\\
\hline
2 &  1, 0 & да\\
\hline
3 &  0, 2 & нет\\
\hline
4 &  3, 1 & да\\
\hline
5 &  1, 2 & нет\\
\hline
6 &  2, 3 & да\\
\hline
\end{tabular}
\end{center}
Вы должны послать ровно один файл, названный \texttt{gondola.cpp}. В этом файле должны быть реализованы функции, описанные выше с указанными прототипами. Все три функции должны быть реализованы, даже если вы не планируете решать все подзадачи. На языках C/C++ вы должны подключить заголовочный файл \texttt{gondola.h}.

\begin{center}
\renewcommand{\arraystretch}{1.5}
\begin{tabular}{|c|c|c|c|}
\hline
\textbf{Подзадача} & \textbf{inputSeq} & \parbox{2.8cm}{\centering \vspace{2mm}\textbf{Возвращаемое значение}\\\vspace{2mm}} & \textbf{Комментарий} \\
\hline
1 & $(1, 2, 3, 4, 5, 6, 7)$ & 1 & --- \\
\hline
1 & $(3, 4, 5, 6, 1, 2)$ & 1 & --- \\
\hline
1 & $(1, 5, 3, 4, 2, 7, 6)$ & 0 & 1 не может быть перед 5 \\
\hline
1 & $(4, 3, 2, 1)$ & 0 & 4 не может быть перед 3 \\
\hline
2 & $(1, 2, 3, 4, 5, 6, 5)$ & 0 & две гондолы с номером 5 \\
\hline
3 & $(2, 3, 4, 9, 6, 7, 1)$ & 1 & последовательность замен (5, 8) \\
\hline
3 & $(10, 4, 3, 11, 12)$ & 0 & 4 не может быть перед 3 \\
\hline
\textbf{Подзадача} & \textbf{gondolaSeq} & \parbox{2.8cm}{\centering \vspace{2mm}\textbf{Возвращаемое значение}\\\vspace{2mm}} & \textbf{replacementSeq} \\
\hline
4 & $(3, 1, 4)$ & 1 & $(2)$ \\
\hline
4 & $(5, 1, 2, 3, 4)$ & 0 & $()$ \\
\hline
5 & $(2, 3, 4, 9, 6, 7, 1)$ & 2 & $(5, 8)$ \\
\hline
\textbf{Подзадача} & \textbf{inputSeq} & \parbox{2.8cm}{\centering \vspace{2mm}\textbf{Возвращаемое значение}\\\vspace{2mm}} & \textbf{Последовательность замен} \\
\hline
7 & $(1, 2, 7, 6)$ & 2 & $(3, 4, 5)$ или $(4, 5, 3)$ \\
\hline
8 & $(2, 3, 4, 12, 6, 7, 1)$ & 1 & $(5, 8, 9, 10, 11)$ \\
\hline
9 & $(4, 7, 4, 7)$ & 0 & $inputSeq$ не является последовательностью гондол \\
\hline
10 & $(3, 4)$ & 2 & $(1, 2)$ или $(2, 1)$ \\
\hline
\end{tabular}
\end{center}

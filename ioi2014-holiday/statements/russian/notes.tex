Вы должны послать ровно один файл, названный \texttt{holiday.cpp}.
В этом файле должна быть реализована функция, описанная выше с указанными ниже
прототипами. На языках C/C++ вы должны подключить заголовочный файл \texttt{holiday.h}.

Обратите внимание, что результат может быть достаточно большим, и функция
findMaxAttraction возвращает 64-битное целое число.

Пусть отпуск Джан-Джи длится 7 дней, количество городов равно 5 (города описаны в таблице
ниже), и он начинает свой отпуск в городе с номером 2. В первый день Джан-Джи посещает 20
достопримечательностей в городе с номером 2. Во второй день Джан-Джи переезжает из
города с номером 2 в город с номером 3, и в третий день он посещает 30
достопримечательностей в городе с номером 3. Следующие три дня Джан-Джи тратит на
переезд из города с номером 3 в город с номером 0 и в седьмой день посещает 10
достопримечательностей в городе с номером 0. Общее количество достопримечательностей,
которые посетил Джан-Джи, составляет $20 + 30 + 10 = 60$, что является максимальным
количеством достопримечательностей, которые он может посетить за 7 дней, начав отпуск в
городе с номером 2.

\begin{center}
\renewcommand{\arraystretch}{1.5}
\begin{tabular}{|c|c|}
\hline
Город & Количество достопримечательностей \\
\hline
0 &  10 \\
\hline
1 &  2 \\
\hline
2 & 20 \\
\hline
3 & 30 \\
\hline
4 & 1 \\
\hline
\end{tabular}
\end{center}

\begin{center}
\renewcommand{\arraystretch}{1.5}
\begin{tabular}{|c|c|}
\hline
День & Действие \\
\hline
1 & посещение достопримечательностей в городе с номером 2 \\
\hline
2 & переезд из города с номером 2 в город с номером 3 \\
\hline
3 & посещение достопримечательностей в городе с номером 3 \\
\hline
4 & переезд из города с номером 3 в город с номером 2 \\
\hline
5 & переезд из города с номером 2 в город с номером 1 \\
\hline
6 & переезд из города с номером 1 в город с номером 0 \\
\hline
7 & посещение достопримечательностей в городе с номером 0 \\
\hline
\end{tabular}
\end{center}
Джан-Джи строит стену из кирпичей одинакового размера. Стена состоит из $n$ столбцов
кирпичей, пронумерованных слева направо от $0$ до $n - 1$. Высотой столбца называется
количество кирпичей в нем. У столбцов могут быть разные высоты.

Джан-Джи строит стену так. Сначала ни в одном из столбцов нет кирпичей. Далее Джан-Джи
выполняет $k$ действий, каждое из которых может быть действием \texttt{добавления} или \texttt{удаления} кирпичей. Строительство считается законченным, когда выполнены все $k$ действий. Перед каждым действием Джан-Джи выбирает интервал из последовательно стоящих столбцов и высоту $h$. После этого он выполняет одно из следующих действий:

\begin{itemize}
\item действие \texttt{добавления}: Джан-Джи добавляет кирпичи в столбцы из выбранного интервала, высота которых меньше чем $h$, так, чтобы она стала равной $h$. Со столбцами, высота которых не меньше, чем $h$, он ничего не делает;
\item действие \texttt{удаления}: Джан-Джи убирает кирпичи из столбцов из выбранного интервала, высота которых больше чем $h$, так, чтобы она стала равной $h$. Со столбцами, высота которых не больше, чем $h$, он ничего не делает.
\end{itemize}

Требуется определить конечную форму стены.

\textbf{Постановка задачи}

По заданному описанию $k$ действий определите количество кирпичей в каждом столбце после
того, как все действия будут выполнены. Вы должны реализовать \texttt{buildWall}.

\begin{itemize} 
\item \texttt{void buildWall(int n, int k, int op[], int left[], int right[],
int height[], int finalHeight[])}
\begin{itemize} 
\item $n$: количество столбцов.
\item $k$: количество действий.
\item $op$: массив длины $k$; $op[i]$ тип $i$-го действия: $1$, если это действие добавления кирпичей, и $2$, если это действие удаления кирпичей, для $0 \le i \le k - 1$.
\item $left$ и $right$: массивы длины $k$; интервал для $i$-го действия начинается со
столбца $left[i]$ и заканчивается столбцом $right[i]$ (включая оба конца
$left[i]$ и $right[i]$); $left[i] \le right[i]$, для $0 \le i \le k - 1$.
\item $height$: массив длины $k$; $height[i]$ высота для $i$-го действия, для $0 \le i \le k - 1$.
\item $finalHeight$: массив длины $n$; вы должны записать конечное количество
кирпичей в столбце $i$ в $finalHeight[i]$, для $0 \le i \le n - 1$.
\end{itemize}
\end{itemize}
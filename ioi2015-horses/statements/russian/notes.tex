Предположим, что прошло три года ($N=3$), и имеется следующая информация:
\begin{center}
\begin{tabular}{|c|c|c|c|}
\hline
  & 0 & 1 & 2\\
\hline
X & 2 & 1 & 3\\
\hline
Y & 3 & 4 & 1\\
\hline
\end{tabular}
\end{center}

При таких начальных значениях Мансур заработает максимальное количество денег, продав
обе лошади в конце года с номером 1. Последовательность его действий такова:

\begin{itemize}
\item Изначально у Мансура одна лошадь.
\item В конце года с номером 0 у него будет две лошади ($1 \cdot X[0] = 2$).
\item В конце года с номером 1 у него все еще будет две лошади ($2 \dot X[1] = 2$ ), сейчас он
может их продать, его выгода составит 8 ($2 \cdot Y[1] = 8$ ).
\end{itemize}

После этого произошло одно уточнение ($M=1$), и значение $Y[1]$ стало равным $2$.

После уточнения имеем следующую информацию:
\begin{center}
\begin{tabular}{|c|c|c|c|}
\hline
  & 0 & 1 & 2\\
\hline
X & 2 & 1 & 3\\
\hline
Y & 3 & 2 & 1\\
\hline
\end{tabular}
\end{center}
В таком случае, одно из оптимальных решений~--- продать одну лошадь в конце года с
номером 0 и три лошади в конце года с номером 2. Последовательность действий такова:

\begin{itemize}
\item Изначально у Мансура одна лошадь.
\item В конце года с номером 0 у него будет две лошади ($1 \cdot X[0] = 2$ ), сейчас он может
продать одну лошадь, его выгода составит 3 ($1 \cdot Y[0] = 3$). У него останется одна
лошадь.
\item В конце года с номером 1 у него будет одна лошадь ($1 \cdot X[1] = 1$ ).
\item В конце года с номером 2 у него будет три лошади ($1 \cdot X[2] = 3$), сейчас он может
продать три лошади, его выгода составит 3 ($3 \cdot Y[2] = 3$). Таким образом,
максимальное суммарное количество денег равно 6 ($3 + 3 = 6$ ).
\end{itemize}
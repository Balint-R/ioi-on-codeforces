В этой задаче нет подзадач. Баллы за решение зависят от количества взвешиваний, сделанных
программой (суммарного количества вызовов функций $getLightest()$, $getHeaviest()$,
$getMedian()$ и/или $getNextLightest()$).

На каждом тесте программе необходимо решить задачу для нескольких наборов монет.
Обозначим за $r$ количество тестов в задаче. Это число зафиксировано тестовыми данными.
Если решение неправильно восстановит порядок хотя бы для одного набора монет хотя бы в
одном тесте, то оно будет оценено в $0$ баллов. Иначе тесты оцениваются по отдельности.

Обозначим за $Q$ минимальное количество взвешиваний, использовав которое возможно
отсортировать любой набор из шести монет с помощью весов Амины. Для усложнения задачи
мы не сообщаем вам число $Q$. Пусть максимальное количество взвешиваний сделанных
вашим решением на всех наборах монет во всех тестах равно $Q + y$ для некоторого целого
числа $y$.

%Пусть максимальное количество взвешиваний, сделанных вашим решением на всех наборах
%монет среди $T$ наборов конкретного теста, равно $Q+x$ для некоторого целого числа $x$.
%(Если вы используете менее $Q$ взвешиваний для всех наборов монет, то $x=0$). В таком %случае, количество баллов за этот тест будет равно $\frac{100}{r((x+y)/5+1)}$, %округленному вниз до двух знаков после запятой.

В таком случае, количество баллов за этот тест будет равно $\frac{100}{r(y / 2.5 + 1)}$, округленному вниз до двух знаков после запятой.


В частности, если ваше решение делает не более $Q$ взвешиваний на каждом наборе монет
каждого теста, вы получите $100$ баллов.

{\it На оригинальном соревновании, система оценивания незначительно отличалась, чтобы наградить решения хорошо работающие в среднем. Здесь это не реализовано }
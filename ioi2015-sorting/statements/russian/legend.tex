У Айжан есть последовательность из $N$ целых чисел $S[0], S[1], \ldots, S[N - 1]$. Эта
последовательность состоит из различных чисел от $0$ до $N - 1$. Айжан пробует
отсортировать эту последовательность по возрастанию, меняя местами некоторые пары
элементов. Ее друг Ермек также собирается менять местами некоторые пары элементов, но
нет уверенности, что он это сделает полезным путем.

Ермек и Айжан собираются изменять последовательность в несколько этапов. На каждом
этапе сначала Ермек совершает обмен элементов местами, затем Айжан делает другой
обмен. Точнее, тот, кто делает обмен, выбирает две корректные позиции и меняет местами
элементы в этих позициях. Заметим, что две позиции не обязательно должны быть
различными. В случае, когда они равны, элемент меняется сам с собой. Такое действие не
изменяет последовательность.

Айжан знает, что Ермек не заботится о том, чтобы отсортировать последовательность $S$. Она
знает какие именно позиции будет выбирать Ермек. Ермек планирует участвовать в $M$ этапах.
Этапы нумеруются от $0$ до $M - 1$. Для всех от $i$ до $M - 1$ включительно, Ермек выберет позиции $X[i]$ и $Y[i]$ в этапе с номером $i$.

Айжан хочет отсортировать последовательность $S$. Перед началом каждого этапа, если
Айжан видит, что последовательность уже отсортирована в возрастающем порядке, весь
процесс прекращается. Дана начальная последовательность $S$ и позиции, которые Ермек
собирается выбрать, требуется найти последовательность обменов, которые сможет
использовать Айжан, чтобы отсортировать последовательность $S$. В некоторых подзадачах
требуется найти как можно более короткую последовательность обменов. Известно, что
последовательность $S$ можно отсортировать за $M$ или меньшее количество этапов.

Если Айжан видит, что последовательность $S$ отсортирована после обмена, сделанного
Ермеком, то она может выбрать обмен в равных позициях (например $0$ и $0$). В результате
последовательность $S$ останется отсортированной после окончания этапа, и Айжан достигнет
цели. Также отметим, если изначально последовательность $S$ отсортирована, то минимальное
количество этапов, необходимое для сортировки, равно $0$.

\textbf{Постановка задачи}

Дана последовательность $S$, количество обменов $M$ и последовательности позиций $X$ и $Y$.
Найдите последовательность обменов, которую Айжан может использовать для сортировки
последовательности $S$. В подзадачах $5$ и $6$ должна быть найдена последовательность обменов
минимальной длины.

Необходимо реализовать функцию \texttt{findSwapPairs}:
\begin{itemize}
\item \texttt{int findSwapPairs(int N, int S[], int M, int X[], int Y[], int P[], int Q[])}~--- эта функция будет вызвана ровно один
раз.
\begin{itemize}
\item $N$~--- длина последовательности $S$.
\item $S$~--- массив целых чисел, содержащий начальную последовательность $S$.
\item $M$~--- количество обменов, запланированных Ермеком. 
\item $X, Y$~--- массивы целых чисел, каждый длиной $M$. На $i$-м ($0 \le i \le M - 1$ ) этапе
Ермек планирует поменять элементы в позициях $X[i]$ и $Y[i]$.
\item $P, Q$~--- массивы целых чисел, в каждом из которых зарезервирована память для
$M$ элементов. Используйте эти массивы для предоставления одной из возможных
последовательностей обменов, которую может использовать Айжан для сортировки
последовательности $S$. Обозначим за $R$ длину последовательности обменов,
найденной решением. Для всех $i$ от $0$ до $R - 1$ включительно, в этапе с номером $i$
Айжан должна выбрать позиции $P[i]$ и $Q[i]$ для получения отсортированной
последовательности $S$.
\item Функция должна возвращать значение $R$(определенное выше).
\end{itemize}
\end{itemize}





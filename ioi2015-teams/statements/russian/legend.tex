В классе $N$ школьников, пронумерованных последовательно целыми числами от $0$ до $N - 1$. На каждый день у учителя есть задания для школьников этого класса. Каждое задание должно быть выполнено командой школьников в тот же день. Задания могут иметь различную сложность. Для каждого задания учитель знает точное количество человек в команде, которая должна его выполнять.

Разные школьники могут предпочитать различные по количеству человек команды. Школьник с номером $i$ может быть включен только в команду с количеством человек от $A[i]$ до $B[i]$ включительно. Каждый день школьник может быть включен не более, чем в одну команду. Некоторые школьники могут быть не включены ни в одну из команд. Каждая команда выполняет ровно одно задание.

Учитель уже выбрал задания для каждого из следующих $Q$ дней. Для каждого из этих дней определите, возможно ли распределить школьников по командам таким образом, что каждое задание выполняется одной командой.

Вам предоставляется описание всех школьников: $N$, $A$ и $B$, а также последовательность из $Q$
 запросов, по одному для каждого дня. Каждый запрос состоит из количества заданий для этого
 дня, обозначенного $M$, и последовательности $K$ длины $M$, элементы которой содержат
 требуемое количество человек в каждой из команд. Для каждого запроса программа должна
 определить, возможно ли сформировать все команды.

Требуется реализовать функции $init$ и $can$:
 \begin{itemize}
 \item $void\ init(int\ N, int\ A[], int\ B[])$~--- эта функция будет вызвана первой и ровно один раз.
 \begin{itemize}
 \item $N$~--- количество школьников в классе.
 \item $A$~--- массив длины $N$, где $A[i]$ задает минимальное количество человек в команде
 для школьника с номером $i$.
 \item $B$~--- массив длины $N$, где $B[i]$ задает максимальное количество человек в
 команде для школьника с номером $i$.
 \item Эта функция не возвращает никакого значения.
 \item Во всех подзадачах $1 \le A[i] \le B[i] \le N$ для всех $i = 0, \ldots, N - 1$.
 \end{itemize}
 \item $int\ can(int\ M, int\ K[])$~--- после однократного вызова $init$, эта функция будет вызвана $Q$ раз
 подряд, по одному разу для каждого дня.
 \begin{itemize}
 \item $M$~--- количество заданий для этого дня.
 \item $K$~---  массив длины $M$, элементы которого содержат требуемое количество человек
 в команде для каждого из этих заданий.
 \item Функция должна возвращать $1$, если возможно сформировать требуемые команды,
 и иначе~--- $0$.
 \item Во всех подзадачах $1 \le M \le N$. Для всех $i = 0, \ldots, M - 1$ выполнено
 $1 \le K[i] \le N$. Заметим, что сумма всех $K[i]$ может превосходить $N$.
 \end{itemize}
 \end{itemize}
 
В примере задания выдаются $2$ дня.Ограничения на
 количество человек в команде для школьников задаются в таблице ниже.
\begin{tabular}{|c|c|c|c|c|}
\hline
Школьник & 0 & 1 & 2 & 3 \\
\hline
$A$ & 1 & 2 & 2 & 2 \\
\hline
$B$ & 2 & 3 & 3 & 4\\\hline
\end{tabular}
В первый день выдается $2$ задания. Требуемое количество человек в командах $1$ и $3$. Эти две
 команды могут быть сформированы включением школьника с номером $0$ в команду из $1$
 человека, а остальных трех школьников в команду из $3$ человек.

 Во второй день тоже выдается $2$ задания, но в этот раз требуемое количество человек в
 командах $1$ и $1$. В этом случае невозможно сформировать команды, так как есть только один
 школьник, который может быть включен в команду из $1$ человека.
 
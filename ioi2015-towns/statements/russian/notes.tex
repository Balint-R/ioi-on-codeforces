Обратите внимание, что номер подзадачи~--- это часть входного файла. Проверяющий модуль
меняет поведение в зависимости от номера подзадачи.

Например, входной файл соответствующий иллюстрации:

\begin{verbatim}
1 1
11
0 17 18 20 17 12 20 16 23 20 11
17 0 23 25 22 17 25 21 28 25 16
18 23 0 12 21 16 24 20 27 24 17
20 25 12 0 23 18 26 22 29 26 19
17 22 21 23 0 9 21 17 26 23 16
12 17 16 18 9 0 16 12 21 18 11
20 25 24 26 21 16 0 10 29 26 19
16 21 20 22 17 12 10 0 25 22 15
23 28 27 29 26 21 29 25 0 21 22
20 25 24 26 23 18 26 22 21 0 19
11 16 17 19 16 11 19 15 22 19 0
\end{verbatim}

Этот формат отличается от задания сети списком дорог. Разрешено изменить пример
проверяющего модуля так, чтобы он использовал другой формат ввода.
Для каждого набора входных данных:
\begin{itemize}
\item $N$~--- целое число от 6 до 110 включительно.
\item Расстояние между любыми двумя городами~--- это целое число от 1 до 1,000,000
включительно.
\end{itemize}

Количество запросов, которые можно сделать в решении, ограничено. Ограничение на
количество запросов различно для разных подзадач и перечислено в таблице ниже. Если
решение превышает этот лимит, то его выполнение будет прервано, и оно будет считаться
выдавшим неправильный ответ на этом тесте.
\begin{center}
\renewcommand{\arraystretch}{1.5}
\begin{tabular}{|c|c|c|c|c|}
\hline
Подзадача & Баллы &  Количество запросов & \parbox{3cm}{\centering \vspace{2mm}Проверка наличия сбалансированного транспортного узла \\\vspace{2mm}} &
Дополнительные ограничения \\
\hline
1 &  13 & $\frac{N(N-1)}{2}$ &  Не нужна & --- \\
\hline
2 & 12 & $\lceil \frac{7N}{2} \rceil$ & Не нужна &  --- \\
\hline
3 & 13 & $\frac{N(N-1)}{2}$& Нужна &  --- \\
\hline
4 & 10 & $\lceil \frac{7N}{2} \rceil$& Нужна & \parbox{6cm}{\centering \vspace{2mm}Каждый мегаполис соединен напрямую с ровно тремя другими населенными пунктами \\\vspace{2mm}} \\
\hline
5 & 13 & $5n$& Нужна &  --- \\
\hline
6 & 39 & $\lceil \frac{7N}{2} \rceil$& Нужна &  --- \\
\hline
\end{tabular}
\end{center}
Обозначение $\lceil x \rceil$ означает минимальное целое число, не меньшее $x$.
Our satellite has just discovered an alien civilization on a remote planet. 
We have already obtained a low-resolution photo of a square area of the planet. 
The photo shows many signs of intelligent life. Our experts have identified 
$n$ points of interest in the photo. The points are numbered from $0$ to
$n - 1$. We now want to take high-resolution photos that contain all of those
$n$ points.

Internally, the satellite has divided the area of the low-resolution 
photo into an $m$ by $m$ grid of unit square cells. Both rows and
columns of the grid are consecutively numbered from $0$ to $m - 1$ 
(from the top and left, respectively). We use $(s, t)$ to denote the
cell in row $s$ and column $t$. The point number $i$ is located 
in the cell $(r_i, c_i)$. Each cell may contain an arbitrary number
of these points.

Our satellite is on a stable orbit that passes directly over the
\textbf{main} diagonal of the grid. The main diagonal is the line
segment that connects the top left and the bottom right corner of 
the grid. The satellite can take a high-resolution photo of any
area that satisfies the following constraints:

\begin{itemize}
\item the shape of the area is a square,
\item two opposite corners of the square both lie on the main diagonal of the grid,
\item each cell of the grid is either completely inside or completely outside the photographed area.
\end{itemize}

The satellite is able to take at most $k$ high-resolution photos.

Once the satellite is done taking photos, it will transmit the 
high-resolution photo of each photographed cell to our home base
(regardless of whether that cell contains some points of interest).
The data for each photographed cell will only be transmitted \textbf{once},
even if the cell was photographed several times.

Thus, we have to choose at most $k$ square areas that will be photographed, 
assuring that:

\begin{itemize}
 \item each cell containing at least one point of interest is photographed 
 at least once, and
 \item the number of cells that are photographed at least once is minimized.
\end{itemize}

Your task is to find the smallest possible total number of photographed cells.

\textbf{Implementation details}

You should implement the following function (method): 

\begin{itemize}
\item \texttt{int64 take\_photos(int n, int m, int k, int[] r, int[] c)}
\begin{itemize}
  \item \texttt{n}: the number of points of interest,
  \item \texttt{m}: the number of rows (and also columns) in the grid,
  \item \texttt{k}: the maximum number of photos the satellite can take,
  \item \texttt{r} and \texttt{c}: two arrays of length $n$ describing the coordinates
  of the grid cells that contain points of interest. 
  For $0 \le i \le n - 1$, the $i$-th point of interest is
  located in the cell $(r[i], c[i])$,
  \item the function should return the smallest possible total number of
  cells that are photographed at least once (given that the photos must cover
  all points of interest).
\end{itemize}
\end{itemize}

Please use the provided template files for details of implementation in your 
programming language.
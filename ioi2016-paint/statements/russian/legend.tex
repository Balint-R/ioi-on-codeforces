Японский кроссворд~--- это известная головоломка. Рассмотрим простую одномерную версию этой головоломки. Японский кроссворд состоит из строки, состоящей из $n$ ячеек. Ячейки пронумерованы от $0$ до $n - 1$ слева направо. Игрок должен покрасить каждую из этих ячеек в черный или белый цвет. Символ '\texttt{X}' используется для обозначения ячейки, покрашенной в черный цвет, и символ '\texttt{\_}' для обозначения ячейки, покрашенной в белый цвет.

Игрок получает последовательность $c = [c_0, \ldots, c_{k - 1}]$, состоящую из $k$ положительных целых чисел,~--- \textit{ключи к разгадке}. Он должен раскрасить ячейки таким образом, чтобы черные ячейки в строке образовывали ровно $k$ последовательных блоков. При этом количество черных ячеек в $i$-м слева блоке (блоки нумеруются с $0$) должно быть равно $c_i$.

Например, если ключи к разгадке $c = [3, 4]$, то разгаданная головоломка должна состоять ровно из двух последовательных блоков черных ячеек: один должен быть длины 3, а второй длины 4. Таким образом, если $n = 10$ и
$c = [3, 4]$, то одним из решений, удовлетворяющих ключам к разгадке, будет "\texttt{\_XXX\_\_XXXX}". Отметим, что решение "\texttt{XXXX\_XXX\_\_}" не удовлетворяет ключам к разгадке, так как блоки черных ячеек не идут в правильном порядке. Кроме того, решение "\texttt{\_\_XXXXXXX\_}" не удовлетворяет ключам к разгадке, так как в этом случае присутствует только один блок черных ячеек, а не два разделенных блока.

Дан частично решенный японский кроссворд. Известны $n$ и $c$, и дополнительно известно, что некоторые ячейки должны быть покрашены в черный, а некоторые в белый цвет. Требуется определить дополнительную информацию о ячейках, то есть найти те ячейки, которые будут покрашены в черный цвет во всех правильных решениях, и те ячейки, которые будут покрашены в белый цвет во всех правильных решениях. Решение является правильным, если оно удовлетворяет ключам к разгадке и согласуется с известными цветами ячеек.

Гарантируется, что для входных данных существует хотя бы одно правильное решение японского кроссворда.

\textbf{Детали реализации}

Вы должны реализовать следующую функцию (метод):

\begin{itemize}

\item \texttt{string solve\_puzzle(string s, int[] c)}

\begin{itemize}
\item \texttt{s}: строка длины $n$. Для всех $i$ ($0 \leq i \leq n - 1$) $i$-й символ равен:
\begin{itemize}
\item '\texttt{X}', если $i$-я ячейка должна быть черной, 
\item '\texttt{\_}', если $i$-я ячейка должна быть белой,
\item '\texttt{.}', если дополнительной информации о $i$-й ячейке не предоставляется.
\end{itemize}
\item \texttt{c}: массив длины $k$, содержащий ключи к разгадке, как описано выше.
\item Функция должна вернуть строку длины $n$. Для всех $i$ $(0 \le i \le n - 1)$ $i$-й символ результирующей строки должен быть равен:
\begin{itemize}
\item '\texttt{X}', если $i$-я ячейка черная во всех правильных решениях,
\item '\texttt{\_}', если $i$-я ячейка белая во всех правильных решениях,
\item '\texttt{?}', иначе (т.е. если существуют два правильных решения головоломки, в одном из которых $i$-я ячейка покрашена в белый цвет, в другом в черный).
\end{itemize}
\end{itemize}
\end{itemize}

ASCII коды символов, используемых в задаче: 
\begin{itemize}
\item '\texttt{X}': 88,
\item '\texttt{\_}': 95, 
\item '\texttt{.}': 46,
\item '\texttt{?}': 63.
\end{itemize}

Пожалуйста, используйте предоставленные шаблоны файлов для уточнения реализации на вашем языке программирования.

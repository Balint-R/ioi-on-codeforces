Anna is working in an amusement park and she is in charge of building the railroad for a new roller coaster. She has already designed $n$ special sections (conveniently numbered from $0$ to $n - 1$) that affect the speed of a roller coaster train. She now has to put them together and propose a final design of the roller coaster. For the purpose of this problem you may assume that the length of the train is zero.

For each $i$ between $0$ and $n - 1$, inclusive, the special section $i$ has two properties:
\begin{itemize}
\item when entering the section, there is a speed limit: the speed of the train must be \textbf{less or equal to} $s_i$ km/h (kilometers per hour),
\item when leaving the section, the speed of the train is \textbf{exactly} $t_i$ km/h, regardless of the speed at which the train entered the section.
\end{itemize}

The finished roller coaster is a single railroad line that contains the $n$ special sections in some order. Each of the $n$ sections should be used exactly once. Consecutive sections are connected with tracks. Anna should choose the order of the $n$ sections and then decide the lengths of the tracks. The length of a track is measured in meters and may be equal to any non-negative integer (possibly zero).

Each meter of the track between two special sections slows the train down by $1$ km/h. At the beginning of the ride, the train enters the first special section in the order selected by Anna, going at $1$ km/h. 

The final design must satisfy the following requirements:
\begin{itemize}
\item the train does not violate any speed limit when entering the special sections;
\item the speed of the train is positive at any moment.
\end{itemize}

In all subtasks except subtask 3, your task is to find the minimum possible total length of tracks between sections. In subtask 3 you only need to check whether there exists a valid roller coaster design, such that each track has zero length.

\textbf{Implementation details}

You should implement the following function (method):
\begin{itemize}
\item \texttt{ int64 plan\_roller\_coaster(int[] s, int[] t)}. 
\begin{itemize}
	\item \texttt{s}: array of length $n$, maximum allowed entry speeds.
	\item \texttt{t}: array of length $n$, exit speeds.
	\item In all subtasks except subtask 3, the function should return the minimum possible total length of all tracks. In subtask 3 the function should return $0$ if there exists a valid roller coaster design such that each track has zero length, and any positive integer if it does not exist.
\end{itemize}
\end{itemize}


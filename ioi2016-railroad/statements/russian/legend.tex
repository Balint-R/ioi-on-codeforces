Анна работает в парке развлечений и занимается проектированием нового аттракциона <<Американские горки>>. Аттракцион представляет собой трассу, по которой будет ездить специальный поезд. Будем считать, что поезд имеет нулевую длину. Анна уже спроектировала $n$ специальных секций (удобно пронумерованных от $0$ до $n - 1$),которые влияют на скорость поезда: подъёмы, резкие торможения, и т. д. Теперь она должна соединить их путями, чтобы получить окончательный план аттракциона.

Для каждого $i$ от $0$ до $n - 1$, включительно, специальная секция $i$
характеризуется двумя значениями:
\begin{itemize}
\item при въезде на эту секцию есть ограничение скорости: скорость поезда при въезде на секцию должна быть \textbf{меньше или равна} $s_i$ км/ч,
\item при выезде с секции, скорость поезда становится равна \textbf{в точности} $t_i$ км/ч, вне зависимости от скорости, с которой поезд въехал на секцию.
\end{itemize}

В итоге план аттракциона должен представлять собой единую трассу, на которой в некотором порядке встречаются все $n$ специальных секций. Каждая
из $n$ секций должна войти в окончательный план аттракциона ровно один раз.
Между последовательными секциями должны быть проложены соединительные пути. Анна должна решить, в каком порядке расположить секции в вдоль трассы аттракциона, и выбрать длину каждого из соединительных путей. Длина каждого соединительного пути измеряется в метрах и должна представлять собой неотрицательное целое число (возможно равное нулю).

Каждый метр соединительного пути между двумя специальными секциями замедляет поезд на $1$ км/ч. В начале поездки поезд въезжает на первую специальную секцию на трассе со скоростью $1$ км/ч.

Окончательный план аттракциона должны отвечать следующим требованиям: 
\begin{itemize}
\item поезд не нарушает ограничения на скорость въезда на специальные секции;
\item скорость поезда должна быть строго положительной в любой момент движения по трассе.
\end{itemize}

Во всех подзадачах, кроме подзадачи 3, требуется выбрать порядок $n$ специальных секций и длины соединительных путей между ними так, чтобы приведенные требования выполнялись, и суммарная длина соединительных путей была как можно меньше. В подзадаче 3 требуется проверить, можно ли спроектировать аттракцион таким образом, чтобы приведенные требования выполнялись, и каждый соединительный путь имел длину равную нулю.

\textbf{Детали реализации}

Вам требуется реализовать следующую функцию (метод): 
\begin{itemize}
\item \texttt{int64 plan\_roller\_coaster(int[] s, int[] t)}
\begin{itemize}
\item \texttt{s}: массив длины $n$, задающий максимальные возможные скорости въезда на специальные секции.
\item \texttt{t}: массив длины $n$, задающий скорости выезда со специальных секций.
\item Во всех подзадачах, кроме подзадачи 3, функция должна возвращать минимальную возможную суммарную длину всех соединительных путей между специальными секциями. В подзадаче 3 функция должна возвращать $0$, если существует такой план аттракциона, что все соединительные пути имеют длину $0$, либо любое положительное число, если такого плана не существует.
\end{itemize}
\end{itemize}

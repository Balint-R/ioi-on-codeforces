\textbf{Пример}

\texttt{plan\_roller\_coaster([1, 4, 5, 6], [7, 3, 8, 6])}

В этом примере даны четыре специальные секции. Оптимально расположить их вдоль трассы в следующем порядке: $0, 3, 1, 2$, и соединить их путями с длинами $1, 2, 0$, соответственно. При этом поезд проезжает по аттракциону следующим образом:
\begin{itemize}
\item Исходно скорость поезда $1$ км/ч.
\item Поезд въезжает на специальную секцию $0$.
\item Поезд выезжает со специальной секции $0$ на скорости $7$ км/ч.
\item Поезд проезжает по соединительному пути длиной $1$ м. В конце
\item соединительного пути его скорость равна $6$ км/ч.
\item Поезд въезжает на специальную секцию $3$ со скоростью $6$ км/ч и
\item выезжает с нее на той же скорости.
\item Выехав с секции $3$, поезд проезжает по соединительному пути длиной $2$м.
\item Его скорость снижается до $4$ км/ч.
\item Поезд въезжает на специальную секцию $1$ на скорости $4$ км/ч и выезжает с нее на скорости $3$ км/ч.
\item Сразу после выезда с секции $1$ поезд въезжает на специальную секцию $2$. Поезд выезжает с секции $2$. Его конечная скорость равна $8$ км/ч.
\end{itemize}

Функция должна вернуть суммарную длину соединительных путей: $1 + 2 + 0 = 3$.
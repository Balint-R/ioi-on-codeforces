Это задача с открытыми тестами. Это означает, что вам не требуется посылать решение, необходимо сдать только ответы на известный заранее набор тестов. Вы можете скачать тесты в секции материалов задачи. Это архив, содержащий файлы 01, 02, 03, \dots. Необходимо сдать один zip-архив содержащий ответы 01.out, 02.out, 03.out, \dots в корне архива. Архив может не содержать ответы на некоторые тесты, в этом случае вы получите вердикт ``\t{Неправильный ответ}'' на этих тестах.

За несколько дней до праздника Навруз (персидский новый год) дедушка пригласил всю
семью в свой сад. Среди гостей будет $k$ детей. Дедушка хочет провести для детей игру в
прятки, чтобы праздник был веселее.

Представим сад в виде прямоугольной сетки размера $m \times n$, состоящей из единичных клеток. Некоторые клетки заняты камнями (возможно, таких клеток нет), а все остальные клетки считаются \texttt{свободными}. Две клетки считаются \texttt{соседними}, если они имеют общую сторону. Таким образом, у каждой клетки может быть до четырёх соседних клеток: две в горизонтальном направлении и две в вертикальном направлении. Дедушка хочет превратить свой сад в лабиринт. Для этого он может посадить кусты в некоторых свободных клетках. Клетки, в которых посажены кусты, перестают быть свободными.

Лабиринт должен удовлетворять следующему требованию. Для каждой пары свободных
клеток $a$ и $b$ должен существовать ровно один \texttt{простой путь} между ними. Простой путь между клетками $a$ и $b$~--- это последовательность свободных клеток, которая начинается с клетки $a$ и заканчивается в клетке $b$, при этом все клетки в ней различны, а также любые две клетки, идущие в последовательности подряд, являются соседними.

Ребёнок может спрятаться в клетке в том и только в том случае, когда эта клетка свободна, и при этом \texttt{ровно} одна из соседних с ней клеток свободна. Никакие два ребёнка не могут прятаться в одной и той же клетке.

В качестве входных данных задана карта сада. Нужно помочь дедушке создать лабиринт, в
котором смогут спрятаться как можно больше детей.

\textbf{Детали реализации}

В данной задаче требуется предоставить только ответы. Также в этой задаче используется
система частичной оценки. Дано $10$ входных файлов, каждый из которых описывает сад
дедушки. Для каждого входного файла необходимо послать на проверку выходной файл с
картой лабиринта. За каждый выходной файл начисляются баллы в зависимости от
количества детей, которые могут спрятаться в этом лабиринте.

В этой задаче не требуется посылать исходный код.





В некоторых случаях поведение проверяющего модуля адаптивно. Это значит, что
проверяющий модуль не использует фиксированной последовательности призов. Вместо этого
ответы от проверяющего модуля могут зависеть от запросов, которые делает ваша программа.
Гарантируется, что проверяющий модуль отвечает таким образом, что после каждого ответа
существует не менее одной последовательности призов, соответствующей всем ответам,
данным до этого.

\begin{center}
\renewcommand{\arraystretch}{1.5}
\begin{tabular}{|c|c|c|}
\hline
Подзадача & Баллы & Дополнительные ограничения на входные данные\\
\hline
1 &  20 & \parbox{13cm}{\centering \vspace{2mm}Среди призов один бриллиант и $n - 1$ леденец (то есть $v = 2$). Вы можете вызвать функцию \texttt{ask} не более $10\,000$ раз. \\\vspace{2mm}} \\
\hline
2 & 80 & Без дополнительных ограничений. \\
\hline
\end{tabular}
\end{center}

В подзадаче 2 вы можете получить часть баллов. Пусть $q$ равно максимальному количеству
вызовов функции \texttt{ask} среди всех тестов в подзадаче. Тогда оценка в этой подзадаче
вычисляется в соответствии со следующей таблицей:
\newcommand{\lt}{\textless}

\begin{center}
\renewcommand{\arraystretch}{1.5}
\begin{tabular}{|c|c|}
\hline
Количество вызовов функции & Результат \\
\hline
$10\,000 \lt q$ & $0$ (вердикт в CMS буде `Wrong Answer') \\
\hline
$6000 \lt q \leq 10\,000$ & $70$ \\
\hline
$5000 \lt q \leq 6000$ & $80 - (q-5000)/100$ \\
\hline
$q \leq 5000$ & $80$ \\
\hline
\end{tabular}
\end{center}



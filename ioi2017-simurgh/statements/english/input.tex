The sample grader reads the input in the following format:
\begin{itemize}
\item line 1: $n$ $m$ ($2 \leq n \leq 500$, $n - 1 \leq m \leq n \cdot (n - 1) / 2$)
\item line $2 + i$ (for all $0 \leq i \leq m - 1$): $u[i]$ $v[i]$ ($0 \leq u[i], v[i] \leq n - 1$)
\item line $2 + m$: $s[0], s[1], \ldots, s[n - 2]$
\end{itemize}

Here, $s[0], s[1], \ldots, s[n - 2]$ are the labels of the royal roads.

For all $0 \leq i \leq m - 1$, road $i$ connects two different cities (i.e., $u[i] \neq v[i]$).

There is at most one road between each pair of cities.

It is possible to travel between any pair of cities through the roads.

The set of all royal roads is a golden set.

\texttt{find\_roads} should call \texttt{count\_common\_roads} at most $q$ times.
In each call, the set of roads specified by $r$ should be a golden set.
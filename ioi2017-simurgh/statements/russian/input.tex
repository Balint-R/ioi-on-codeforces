Пример проверяющего модуля читает входные данные в следующем формате:
\begin{itemize}
\item строка 1: $n$ $m$ ($2 \leq n \leq 500$, $n - 1 \leq m \leq n \cdot (n - 1) / 2$)
\item строка $2 + i$ (для всех $0 \leq i \leq m - 1$): $u[i]$ $v[i]$ ($0 \leq u[i], v[i] \leq n - 1$)
\item line $2 + m$: $s[0], s[1], \ldots, s[n - 2]$
\end{itemize}

Здесь $s[0], s[1], \ldots, s[n - 2]$ обозначают номера шахских дорог.

Для всех $0 \leq i \leq m - 1$, дорона $i$ соединяет два различных города (то есть $u[i] \neq v[i]$).

Между каждой парой городов есть не более одной дороги.

Используя дороги, возможно проехать между каждой парой городов.

Набор всех шахских дорог является золотым.

\texttt{find\_roads} должен вызвать \texttt{count\_common\_roads} не более $q$ раз. В каждом вызове набор дорог, задаваемых массивом $r$, должен являться золотым.
Арезу и Борзу~--- близнецы. На день рождения им подарили восхитительную игрушечную
железную дорогу, из которой они построили железнодорожную сеть, состоящую из $n$ станций и $m$ односторонних путей. Станции пронумерованы от $0$ до $n - 1$. Каждый путь выходит из
одной станции и приходит либо к той же станции, либо к некоторой другой. С каждой станции
выходит хотя бы один путь.

Некоторый станции являются \texttt{зарядными}. Когда поезд прибывает на зарядную станцию, он полностью заряжается. Энергии полностью заряженного поезда хватает, чтобы проехать $n$
путей без дополнительной зарядки. Иными словами, поезд полностью разряжается в момент
перехода на $n + 1$-й по счёту путь с момента последней зарядки.

На каждой станции имеется переключатель, который может указывать на любой из путей,
выходящий от станции. Поезд может покинуть станцию только по пути, на который указывает
переключатель.

Близнецы хотят сыграть в игру, используя свой поезд. Они поделили между собой все станции:
каждая станция принадлежит либо Арезу, либо Борзу. У них есть один поезд. В начале игры
полностью заряженный поезд находится на станции $s$. Игра начинается с того, что владелец
станции $s$ устанавливает переключатель на один из исходящих путей. После этого дети
включают поезд и он начинает двигаться вдоль путей.

Когда поезд первый раз оказывается на некоторой станции, владелец станции устанавливает
переключатель, расположенный на этой станции. Установленный переключатель остаётся в
том же положении до конца игры. Таким образом, если поезд оказывается на станции, где он
уже был, то он отправляется с неё по тому же пути, что и раньше.

Так как станций конечное количество, рано или поздно поезд попадёт в \texttt{цикл}. Цикл это такая последовательность \texttt{различных} станций $c[0], c[1], \ldots, c[k - 1]$, что со станции $c[i]$ (для $0 \le i < k - 1$) поезд выезжает по пути до станции $c[i + 1]$, а со станции $c[k - 1]$ поезд выезжает по пути до станции $c[0]$. Обратите внимание, что цикл может состоять из единственной станции (то есть, $k = 1$), если поезд со станции $c[0]$ выезжает по пути, ведущему обратно на станцию $c[0]$.

Арезу выигрывает игру, если поезд продолжает движение неограниченно долгое время, а
Борзу выигрывает, если поезд рано или поздно разрядится. Иными словами, если среди
станций $c[0], c[1], \ldots, c[k - 1]$ есть хотя бы одна зарядная станция, то поезд сможет постоянно заряжаться и неограниченно долго перемещаться вдоль данного цикла, и Арезу выигрывает. В противном случае поезд разрядится (возможно, после некоторого количества полных перемещений вдоль цикла), и выигрывает Борзу.

Задано описание железнодорожной сети. Арезу и Борзу собираются провести $n$ игр. В $s$-й игре для $0 \le s \le n - 1$ поезд исходно находится на станции $s$. Для каждой из игр требуется определить, есть ли у Арезу стратегия, гарантирующая ей выигрыш вне зависимости от ходов Борзу.

\textbf{Детали реализации}

Требуется реализовать следующую функцию (метод):

\begin{itemize}
\item \texttt{int[] who\_wins(int[] a, int[] r, int[] u, int[] v)}
\begin{itemize}
\item $a$: массив длины $n$. Если станция $i$ принадлежит Арезу, то $a[i] = 1$. В противном случае станция $i$ принадлежит Борзу и $a[i] = 0$.
\item $r$: массив длины $n$. Если станция $i$ является зарядной, то $r[i] = 1$. В противном случае будет выполнено $r[i] = 0$.
\item $u$ and $v$:  массивы длины $m$, задающие для каждого $0 \leq i \leq m - 1$, tодносторонний путь, ведущий от станции $u[i]$ к станции $v[i]$.
\item функция должна возвращать массив $w$ длины $n$. Для каждого $0 \leq i \leq n-1$ значение $w[i]$ должно равняться $1$, если Арезу выигрывает в игре, начинающейся со
станции $i$, вне зависимости от ходов Борзу. В противном случае значение $w[i]$ должно
быть установлено в $0$.
\end{itemize}
\end{itemize}








\newcommand{\gt}{\textgreater} 
\newcommand{\lt}{\textless} 


\begin{center}
\renewcommand{\arraystretch}{1.5}
\begin{tabular}{|c|c|c|}
\hline
Подзадача & Баллы & \parbox{10cm}{\centering \vspace{2mm}Дополнительные ограничения на входные данные\\\vspace{2mm}} \\
\hline
1 & 2 & \parbox{10cm}{\centering \vspace{2mm}Каждое целое число $i$ ($1\le i \le M$) встречается в последовательности $A_0,A_1,\ldots A_{N-1}$ не более одного раза.\\\vspace{2mm}}\\
\hline
2 & 4 & \parbox{10cm}{\centering \vspace{2mm}Каждое целое число $i$ ($1\le i \le M$) встречается в последовательности $A_0,A_1,\ldots A_{N-1}$ не более двух раз.\\\vspace{2mm}}\\
\hline
3 & 10 &\parbox{10cm}{\centering \vspace{2mm} Каждое целое число $i$ ($1\le i\le M$) встречается в последовательности $A_0,A_1,\ldots A_{N-1}$ не более четырех раз.\\\vspace{2mm}} \\
\hline
4 & 10 & $N=16$ \\
\hline
5 & 18 & $M=1$ \\
\hline
6 & 56 & Нет дополнительных ограничений \\
\hline
\end{tabular}
\end{center}
Для каждого теста, если ваша программа получает на нём вердикт \textbf{Accepted}, ваши
баллы за этот тест вычисляются по следующим правилам в зависимости от $S$:
\begin{itemize}
    \item Если $S\le N + \log_2{N}$, вы получаете полный балл за этот тест.
    \item Для каждого теста в подзадачах $5$ и $6$, если $N + \log_2{N} \lt S \le 2\cdot N$, вы
получаете частичные баллы. Баллы за тест равны значению $0.5 + 0.4 \times (\frac{2N - S}{N - \log_2{N}})^2$, умноженному на балл для этой подзадачи.
\item В противном случае баллы равны $0$.
\end{itemize}
Баллы за каждую подзадачу равны минимуму баллов за тест в этой подзадаче.



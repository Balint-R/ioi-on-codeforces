In Japan, cities are connected by a network of highways. This network consists of $N$ cities and $M$ highways. Each highway connects a pair of distinct cities. No two highways connect the same pair of cities. Cities are numbered from $0$ through $N-1,$ and highways are numbered from $0$ through $M-1.$ You can drive on any highway in both directions. You can travel from any city to any other city by using the highways.

A toll is charged for driving on each highway. The toll for a highway depends on the \bf{traffic} condition on the highway. The traffic is either \bf{light} or \bf{heavy}. When the traffic is light, the toll is $A$ yen (Japanese currency). When the traffic is heavy, the toll is $B$ yen. It's guaranteed that $A < B.$ Note that you know the values of $A$ and $B.$

You have a machine which, given the traffic conditions of all highways, computes the smallest total toll that one has to pay to travel between the pair of cities $S$ and $T (S \neq T),$ under the specified traffic conditions.

However, the machine is just a prototype. The values of $S$ and $T$ are fixed (i.e., hardcoded in the machine) and not known to you. You would like to determine $S$ and $T.$ In order to do so, you plan to specify several traffic conditions to the machine, and use the toll values that it outputs to deduce $S$ and $T.$ Since specifying the traffic conditions is costly, you don't want to use the machine many times.

\bf{Implementation details}

You should implement the following procedure:

\t{find_pair(int N, int[] U, int[] V, int A, int B)}

\begin{itemize}
\item $N$: the number of cities. 
\item $U$ and $V$: arrays of length $M,$ where $M$ is the number of highways connecting cities. For each $i : 0 \le i \le M-1,$ the highway $i$ connects the cities $U[i]$ and $V[i].$ 
\item $A$: the toll for a highway when the traffic is light. 
\item $B$: the toll for a highway when the traffic is heavy. 
\item This procedure is called exactly once for each test case. 
\item Note that the value of $M$ is the lengths of the arrays, and can be obtained as indicated in the implementation notice.
\end{itemize}

\t{int64 ask(int[] w)}

\begin{itemize}
\item The length of $w$ must be $M.$ The array $w$ describes the traffic conditions. 
\item For each $i: 0 \le i \le M-1, w[i]$ gives the traffic condition on the highway $i.$ The value of $w[i]$ must be either $0$ or $1.$ 
\begin{itemize}
\item $w[i] = 0$ means the traffic of the highway $i$ is light. 
\item $w[i] = 1$ means the traffic of the highway $i$ is heavy. 
\end{itemize}
\item This function returns the smallest total toll for travelling between the cities $S$ and $T,$ under the traffic conditions specified by $w.$ 
\item This function can be called at most $100$ times (for each test case).
\end{itemize}

\t{find_pair} should call the following procedure to report the answer:

\t{answer(int s, int t)}

\begin{itemize}
\item \t{s} and \t{t} must be the pair $S$ and $T$ (the order does not matter). 
\item This procedure must be called exactly once.
\end{itemize}

If some of the above conditions are not satisfied, your program is judged as \bf{Wrong Answer}. Otherwise, your program is judged as \bf{Accepted} and your score is calculated by the number of calls to \t{ask} (see Subtasks).

\bf{Constraints}

\begin{itemize}
\item $2 \le N \le 90\,000$
\item $1 \le M \le 130\,000$
\item $1 \le A < B \le 1\,000\,000\,000$
\item For each $i: 0 \le i M-1$
\begin{itemize}
\item $0 \le U[i] \le N-1$
\item $0 \le V[i] \le N-1$
\item $U[i] \neq V[i]$
\end{itemize}
\item $(U[i], V[i]) \neq (U[j], V[j])$ and  $(U[i], V[j]) \neq (V[j], U[j])$ for all $i,j: 0 \le i < j \le M-1$
\item You can travel from any city to any other city by using the highways.
\item $0 \le S \le N-1$
\item $0 \le T \le N-1$
\item $S \neq T$
\end{itemize}

In this problem, the grader is NOT adaptive. This means that $S$ and $T$ are fixed at the beginning of the running of the grader and they do not depend on the queries asked by your solution.

\bf{Sample grader}

The sample grader reads the input in the following format:

\begin{tabular}{lllll}
line&$1$&:&$N$ $M$ $A$ $B$ $S$ $T$&\\
line&$2+i$&:&$U[i]$ $V[i]$&$(0 \le i \le M-1)$\\
\end{tabular}

If your program is judged as \bf{Accepted}, the sample grader prints \t{Accepted: q}, with $q$ the number of calls to \t{ask}.

If your program is judged as \bf{Wrong Answer}, it prints \t{Wrong Answer: MSG}, where \t{MSG} is one of:

\begin{itemize}
\item \t{answered not exactly once}: The procedure answer was not called exactly once. 
\item \t{w is invalid}: The length of $w$ given to \t{ask} is not $M$ or $w[i]$ is neither $0$ nor $1$ for some $i: 0 \le i \le M-1.$ 
\item \t{more than 100 calls to ask}: The function \t{ask} is called more than $100$ times. 
\item \t{{s, t} is wrong}: The procedure \t{answer} is called with an incorrect pair $s$ and $t.$
\end{itemize}

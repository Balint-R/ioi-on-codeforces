$N$ гор расположены в ряд и пронумерованы целыми числами от $0$ до $N-1$ слева
направо. Высота горы с номером $i$ равна $H_i$ ($0\le i\le N-1$). На вершине каждой
горы живёт ровно один человек.

Вы собираетесь организовать $Q$ встреч, пронумерованных целыми числами от $0$ до
$Q-1$. Во встрече с номером $j$ ($0 \le j 
\le Q-1$) будут участвовать все люди,
живущие на горах с номерами от $L_j$ до $R_j$ включительно ($0\le L_j \le R_j \le N-1$). В
качестве места встречи вам необходимо выбрать номер $x$ ($L_j \le x \le R_j$) горы, на
которой произойдёт встреча. Стоимость такой встречи, в зависимости от выбора
горы, вычисляется следующим образом:
\begin{itemize}
    \item Стоимость встречи для участника, который живет на горе с номером
$y$ ($L_j \le y \le R_j$), равна максимальной высоте среди гор с номерами между $x$ и $y$
включительно. В частности, стоимость встречи для участника с горы $x$ равна $H_x$~--- высоте горы $x$.

\item Стоимость встречи равна сумме стоимостей встречи для каждого из её
участников.
\end{itemize}


Для каждой встречи вам необходимо найти минимальную возможную стоимость
её проведения.

Обратите внимание, что после встречи все участники возвращаются обратно на
вершины гор, где они живут. Таким образом, стоимость встречи не зависит от
предыдущих встреч.


\textbf{Детали реализации}

Вам следует реализовать одну функцию :

\begin{itemize}
    \item \texttt{int64[] minimum\_costs(int[] H, int[] L, int[] R)}
    \begin{itemize}
        \item $H$: массив длины $N$, описывающий высоты гор.
        \item $L$ и $R$: массивы длины $Q$, описывающие диапазоны участников встреч.
        \item Эта функция должна возвращать массив $C$ длины $Q$. Значение $C_j$ ($0\le j\le Q-1$)
должно равняться минимальной возможной стоимости
проведения встречи с номером $j$.
\item Обратите внимание, что значения $N$ и $Q$ являются длинами массивов и могут быть получены, как указано в памятке о деталях реализации.
    \end{itemize}
\end{itemize}



Вы планируете проводить международное соревнование по программированию в прямоугольном зале, в котором $ H \cdot W $ посадочных мест, они организованы в виде прямоугольника, содержащего $ H $ строк и $ W $ столбцов. Строки пронумерованы от $ 0 $ до $ H-1 $ , а столбцы пронумерованы от $ 0 $ до $ W-1 $ . Посадочное место в строке $ r $ и столбце $ c $ обозначается как $(r, c).$ Вы пригласили $ H \cdot W $ участников, пронумерованных от $ 0 $ до $ H \cdot W-1 . $ Вы также подготовили распределение участников по посадочным местам, в соответствии с которым участник с номером $ i $ $( 0 \le i \le H \cdot W-1 ) $ занимает посадочное место $ (R_i, C_i). $ В соответствии с распределением на каждом посадочном месте размещается ровно один участник. 

Множество $ S $ посадочных мест в зале называется \bf{прямоугольником}, если для некоторых целых чисел $ r_1, r_2, c_1, $ и $ c_2 $ выполнены следующие условия:

\begin{itemize}
\item $ 0 \le r_1 \le r_2 \le H-1 $ 
\item $ 0 \le c_1 \le c_2 \le W-1 $ 
\item $ S $ состоит в точности из всех посадочных мест $ (r, c) $ для которых $ r_1 \le r \le r_2 $ и $ c_1 \le c \le c_2 $ . 
\end{itemize}

Прямоугольник, состоящий из $ k $ посадочных мест $ ( 1 \le k \le H \cdot W), $ называется \bf{красивым}, если на местах из этого прямоугольника размещаются в точности участники с номерами от $ 0 $ до $ k-1. $ \bf{Красотой} распределения участников по посадочным местам называется количество красивых прямоугольников для данного распределения.
После подготовки вашего распределения участников по местам, вы выполняете несколько запросов обмена двух участников посадочными местами. А именно, дано $ Q $ таких запросов, пронумерованных от $ 0 $ до $ Q-1 $ в хронологическом порядке. Запрос с номером $j$ $(0 \le j \le Q-1)$ состоит том, что участники с номерами $ A_j $ и $ B_i $ меняются посадочными местами. Вы немедленно обрабатываете каждый запрос и обновляете распределение участников по посадочным местам. После каждого обновления вам требуется вычислить красоту текущего распределения участников по посадочным местам.

\bf{Детали реализации}

Вам необходимо реализовать следующие процедуры и функции:

\t{give_initial_chart(int H, int W, int[] R, int[] C)}

\begin{itemize}
\item $ H, W: $ количество строк и количество столбцов. 
\item $ R, C: $ массивы длины $ H \cdot W $ , задающие исходное распределение участников по посадочным местам.
\item Процедура будет вызвана ровно один раз до любого вызова \t{swap_seats}.
\end{itemize}

\t{int swap_seats(int a, int b) }

\begin{itemize}
\item Эта функция описывает запрос обмена двух участников посадочными местами.
\item $ a, b: $ участники, которые меняются посадочными местами. 
\item Эта функция будет вызвана $Q$ раз. 
\item Функция должна вернуть красоту распределения участников по посадочным местам после обмена
\end{itemize}

\bf{Ограничения }

\begin{itemize}
\item $ 1 \le H $ 
\item $ 1 \le W $ 
\item $ H \cdot W \le 1\,000\,000 $ 
\item $ 0 \le Ri \le H-1 $ для всех $ i: 0 \le i \le H \cdot W-1 $ 
\item $ 0 \le Ci \le W-1 $ для всех $ i: 0 \le i \le H \cdot W-1 $ 
\item $ (R_i, C_i) \neq (R_j, C_j) $ для всех $ i, j: 0 \le i < j \le H \cdot W-1 $ 
\item $ 1 \le Q \le 50\,000 $ 
\item $ 0 \le a \le H \cdot W -1 $ для всех вызовов \t{swap_seats}
\item $ 0 \le b \le H \cdot W -1 $ для всех вызовов \t{swap_seats}
\item $ a \neq b $ для всех вызовов \t{swap_seats}
\end{itemize}

\bf{Пример проверяющего модуля}

Пример проверяющего модуля считывает входные данные в следующем формате:

\begin{tabular}{ccccc}
строка& $ 1 $ &:& $H$ $W$ $Q$ &\\
строка& $ 2+i $ &:& $R_i$ $C_i$ & $ (0 \le i \le H \cdot W-1) $ \\
строка& $ 2+H \cdot W+j $ &:& $A_j$ $B_j$ & $ (0 \le j \le Q-1) $ \\
\end{tabular}

Здесь $ A_j $ и $ B_j $ --- параметры вызова \t{swap_seats} для запроса $ j.$

Пример проверяющего модуля выводит ваши ответы в следующем формате:

\begin{tabular}{ccccc}
строка& $ 1+j $ &:&возвращаемое значение \t{swap_seats} для запроса $ j $ & $ (0 \le j \le Q-1) $ 
\end{tabular}

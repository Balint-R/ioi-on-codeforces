Это output-only задача. Это значит, что вам не нужно отправлять свое решение, а нужно только отправлять выходные данные для заданного набора входных данных. Вы можете скачать тестовые данные в разделе проблемных материалов. Это архив, содержащий входные файлы 01, 02, 03, \dots. Вам необходимо отправить один zip-архив, содержащий ваши выходы 01.out, 02.out, 03.out, \dots в корне архива. В архиве может не быть ответов на некоторые тесты, в этих тестах вы получите вердикт <<Неверный ответ>>.

Азербайджан известен своими коврами. Вы мастер рисунка по коврам, и вы хотите оформить новый ковёр с использованием \texttt{ломаной}. Ломаная - это последовательность из $t$ отрезков на двумерной плоскости, которая задаётся последовательностью из $t+1$ вершин $p_0, \ldots, p_t$ следующим образом. Для каждого $0 \leq j \leq t-1$ между вершинами $p_j$ и $p_{j+1}$ проводится отрезок.

Приступая к рисунку, вы отметили $n$ \texttt{точек} на плоскости. Точка $i$ ($1 \leq i \leq n$) имеет координаты $(x[i], y[i])$. \texttt{Не существует пары точек с совпадающими $x$-координатами или с совпадающими $y$-координатами.}

Теперь вы хотите найти последовательность вершин $(sx[0], sy[0]), (sx[1], sy[1]), \ldots, (sx[k], sy[k])$, задающую ломаную со следующими свойствами:

\begin{itemize}
\item Ломаная начинается в $(0,0)$ (то есть, $sx[0]=0$ и $sy[0]=0$ ),
\item Ломаная проходит через все точки (точки не обязаны совпадать с вершинами ломаной),
\item Ломаная состоит только из горизонтальных и вертикальных отрезков (у любых двух последовательных вершин ломаной должны совпадать либо х-координаты, либо у-координаты).
\end{itemize}

Ломаная может самопересекаться и самонакладываться произвольным образом. Формально говоря, любая точка плоскости может содержаться в произвольном количестве отрезков ломаной.

Это задача с открытыми тестами и частичной системой оценивания каждого теста. Вам дано $10$ входных файлов, описывающих положение точек. Для каждого входного файла вы должны отправить выходной файл, описывающий ломаную с требуемыми свойствами. Ваши баллы за каждый выходной файл зависят от \texttt{количества отрезков} в найденной ломаной.

В данной задаче не требуется отправлять какой-либо исходный код на проверку.
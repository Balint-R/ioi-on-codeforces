В начале 19-го века правитель Хусейн-Кули-хан приказал построить дворец на
плато неподалёку от реки. Представим плато в виде клетчатого прямоугольника
$n \times m$. Строки прямоугольника пронумерованы числами от $0$ до $n-1$, столбцы
пронумерованы числами от $0$ до $m-1$. Будем обозначать клетку в $i$-й строке и $j$-м
столбце ($0 \leq i \leq n-1, 0 \leq j \leq m-1$) за $(i,j)$. Каждая клетка обладает своей
высотой, обозначим высоту клетки $(i,j)$ за  $a[i][j]$.

Для постройки дворца Хусейн-Кули-хан попросил архитекторов выбрать
\texttt{прямоугольную область}. В область не должна попасть ни одна из клеток на
границе клетчатого прямоугольника (то есть, из строки $0$, строки $n-1$, столбца $0$
и столбца $m-1$). Таким образом, архитекторы должны выбрать четыре целых
числа $r_1$, $r_2$, $c_1$, и $c_2$ ($1 \leq r_1 \leq r_2 \leq n-2$ и $1 \leq c_1 \leq c_2 \leq m-2$), задающие область, состоящую из всех клеток $(i, j)$, таких что и $r_1 \leq i \leq r_2$ и $c_1 \leq j \leq c_2$.

Дополнительно, область называется \texttt{допустимой} тогда и только тогда, когда для
любой клетки $(i, j)$ из области выполнено следующее условие:
\begin{itemize}
\item Рассмотрим две клетки, прилегающие к области в строке $i$ (клетки $(i, c_1-1)$ и $(i, c_2+1)$) и две клетки, прилегающие к области в столбце $j$ (клетки $(r_1-1, j)$ и $(r_2+1, j)$).
Высота клетки $(i,j)$ должна быть строго меньше, чем высоты
всех этих четырёх клеток.
\end{itemize}

Ваша задача помочь архитекторам определить число допустимых областей для
дворца (то есть, количество способов выбрать четвёрку чисел $r_1$, $r_2$, $c_1$ и $c_2$ 
определяющую допустимую область).

\texttt{Детали реализации}

Вы должны реализовать следующую функцию:

\begin{itemize}
\item \texttt{int64 count\_rectangles(int[][] a)}
\begin{itemize}
\item $a$: двумерный массив $n$ на $m$, состоящий из целых чисел, обозначающих
высоты клеток.
\item Функция должна вернуть количество допустимых областей для постройки
крепости.
\end{itemize}
\end{itemize}


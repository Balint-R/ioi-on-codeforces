В Баку $n$ достопримечательностей, которые пронумерованы от $0$ до $n-1$, а также $m$
двусторонних дорог, которые пронумерованы от $0$ до $m-1$. Каждая из дорог
соединяет две различные достопримечательности. По дорогам можно добраться
от любой достопримечательности до любой другой.

Фатима планирует посетить все достопримечательности за три дня. Она хочет
посетить $a$ достопримечательностей в первый день, $b$ достопримечательностей во
второй день и $c$ достопримечательностей в третий день. Таким образом, она
собирается разделить $n$ достопримечательностей на три множества $A$, $B$ и $C$
размеров $a$, $b$ и $c$ соответственно. Каждая достопримечательность должна
принадлежать ровно одному из множеств, поэтому $a + b + c = n$.

Фатима хочет найти такие множества $A$, $B$ и $C$, что \texttt{хотя бы два} из этих трёх
множеств являются \texttt{связными}. Множество достопримечательностей $S$
называется связным, если между любыми двумя достопримечательностями из $S$
можно переместиться по дорогам, не посещая достопримечательности, которые
не лежат в $S$. Разбиение на множества $A$, $B$ и $C$ называется \texttt{корректным}, если
оно удовлетворяет всем условиям, описанным выше.
Помогите Фатиме найти корректное разбиение достопримечательностей для
данных чисел $a$, $b$ и $c$, либо определите, что ни одного корректного разбиения не
существует. Если существует несколько корректных разбиений, вы можете найти
любое из них.

\texttt{Детали реализации}

Вы должны реализовать следующую функцию:

\begin{itemize}
\item \texttt{int[] find\_split(int n, int a, int b, int c, int[] p, int[] q)}
\begin{itemize}
\item $n$: количество достопримечательностей.
\item $a$, $b$, и $c$: требуемые размеры множеств $A$, $B$, и $C$.
\item $p$ и $q$: массивы длины $m$, содержащие концы дорог. Для каждого $i$ ($0 \leq i \leq m-1$), $p[i]$ и $q[i]$ описывают две достопримечательности,
соединённые дорогой $i$.
\item Функция должна вернуть массив длины $n$. который обозначим $s$.
Если корректного разбиения не существует, $s$ должен содержать $n$ нулей.
В противном случае, для $0 \leq i \leq n-1$, $s[i]$ должно быть равно $1$, $2$ или $3$, если достопримечательность $i$ должна принадлежать множеству $A$, $B$ или $C$ соответственно.
\end{itemize}
\end{itemize}




Вам необходимо реализовать программу компьютерного зрения для робота.
Каждый раз, когда камера робота делает снимок, он сохраняется в памяти робота
как чёрно-белое изображение. Каждое изображение представляет собой
прямоугольную таблицу пикселей размером $H \times W$, строки пронумерованы от $0$
до $H-1$, а столбцы пронумерованы от $0$ до $W-1$. В каждом изображении \texttt{ровно
два} чёрных пикселя, остальные пиксели белые.

Робот может обрабатывать каждое изображение, используя программу,
составленную из простых инструкций. Вам даны значения $H$, $W$ и положительное
целое число $K$. Ваша цель~--- написать функцию, которая составит программу для
робота, определяющую для любого изображения, верно ли, что \texttt{расстояние}
между двумя чёрными пикселями равно в точности $K$. Здесь расстояние между
пикселем в строке $r_1$ в столбце $c_1$ и пикселем в строке $r_2$ в столбце $c_2$ равно $|r_1-r_2|+|c_1-c_2|$. В этой формуле $|x|$ обозначает абсолютную величину числа $x$,
которая равна $x$, если $x \geq 0$, либо $-x$, если $x < 0$.

Рассмотрим, как работает робот.

Память робота представляет собой достаточно большой массив ячеек,
проиндексированных, начиная с $0$. Каждая ячейка может содержать $0$ или $1$, и
значения ячеек после того, как они присваиваются, не могут быть изменены.
Изображение сохранено в памяти строка за строкой в ячейках с $0$ по $H \cdot W - 1$.
Первая строка сохраняется в ячейках с $0$ по $W-1$, а последняя строка
сохраняется в ячейках с $(H - 1) \cdot W$ по $H \cdot W - 1$. Таким образом, если пиксель в строке $i$ в столбце $j$ чёрный, то значение ячейки $i \cdot W + j$ равно $1$, иначе оно
равно $0$.

Программа для робота представляет собой последовательность \texttt{инструкций},
пронумерованных последовательными целыми числами, начиная с $0$. Когда
программа запускается, инструкции исполняются одна за другой. Каждая
инструкция читает значения одной или более ячеек (будем называть эти значения
\texttt{входами} инструкции) и результат её выполнения представляет собой одно число,
равное $0$ или $1$ (будем называть это значение \texttt{результатом} инструкции).
Результат инструкции $i$ сохраняется в ячейке $H \cdot W + i$. Входами для инструкции $i$
могут быть только значения ячеек, в которых содержатся пиксели изображения
или результаты предыдущих инструкций, то есть ячейки с номерами от $0$ до $H \cdot W + i - 1$.


Есть четыре типа инструкций:
\begin{itemize}
\item \texttt{NOT}: имеет ровно один вход. Её результат равен $1$, если вход равен $0$, иначе результат равен $0$.
\item \texttt{AND}: имеет один или более входов. Её результат равен $1$, если и только если \texttt{все} её входы равны $1$.
\item \texttt{OR}: имеет один или более входов. Её результат равен $1$, если и только если
\texttt{хотя бы один} её вход равен $1$.
\item \texttt{XOR}: имеет один или более входов. Её результат равен $1$, если и только если \texttt{нечётное количество} её входов равны $1$.
\end{itemize}

Результат последней инструкции в программе должен быть равен $1$, если
расстояние между чёрными пикселями равно в точности $K$, либо $0$ в противном
случае.


\textbf{Детали реализации}

Вы должны реализовать следующую функцию:

\begin{itemize}
\item \texttt{void construct\_network(int H, int W, int K)}
\begin{itemize}
\item $H, W$: размеры изображения, полученного камерой робота
\item $K$: положительное число
\item Функция должна создать программу для робота. Для любого изображения,
сделанного камерой робота, созданная программа должна определять, верно
ли, что расстояние между двумя черными пикселями на изображении равно в
точности $K$.
\end{itemize}
\end{itemize}

Эта функция должна вызвать одну или более из следующих функций, каждая из
которых добавляет инструкцию в конец программы робота (которая исходно
пуста):

\begin{verbatim}
int add_not(int N)
int add_and(int[] Ns)
int add_or(int[] Ns)
int add_xor(int[] Ns)
\end{verbatim}

\begin{itemize}
\item Функции добавляют инструкцию \texttt{NOT}, \texttt{AND}, \texttt{OR} и \texttt{XOR}, соответственно.
\item $N$ (for \texttt{add\_not}): номер ячейки, в которой находится вход для
инструкции \texttt{NOT}
\item $Ns$ (для функций \texttt{add\_and}, \texttt{add\_or}, \texttt{add\_xor}): массив, содержащий номера ячеек, которые содержат входы для инструкции \texttt{AND}, \texttt{OR}, или \texttt{XOR}
\item Каждая функция возвращает номер ячейки, в которую помещается результат
выполнения инструкции. Последовательные вызовы функций возвращают последовательные целые числа, начиная с $H \cdot W$.
\end{itemize}

Программа робота может содержать не более $10\,000$ инструкций. Инструкции
могут суммарно использовать не более $1\,000\,000$ входных значений. Другими
словами, суммарная длина массивов $Ns$ во всех вызовах \texttt{add\_and}, \texttt{add\_or} и \texttt{add\_xor} плюс количество вызовов \texttt{add\_not} не должно превышать $1\,000\,000$.

После добавления последней инструкции функция \texttt{construct\_network} должна
завершить работу. Программа для робота будет протестирована на некотором
множестве изображений. Ваше решение проходит тест, если для каждого из этих
изображений результат последней инструкции равен $1$ тогда и только тогда, когда
расстояние между чёрными пикселями на изображении в точности равно $K$.

По итогам тестирования вашего решения вы можете получить следующие
сообщения об ошибках на английском языке:

% <!-- WARNING: DO NOT TRANSLATE THE MESSAGES IN TRIPLE QUOTES BELOW -->
\begin{itemize}
\item \texttt{Instruction with no inputs}: функция  \texttt{add\_and}, \texttt{add\_or}, или \texttt{add\_xor} получила в качестве аргумента пустой массив.
\item \texttt{Invalid index}: некорректный (возможно, отрицательный) номер ячейки
указан как вход для \texttt{add\_and}, \texttt{add\_or}, \texttt{add\_xor}, или \texttt{add\_not}.
\item \texttt{Too many instructions}: ваша функция попыталась добавить в программу
более $10\,000$ инструкций.
\item \texttt{Too many inputs}: инструкции суммарно используют более чем $1\,000\,000$
входных значений.
\end{itemize}




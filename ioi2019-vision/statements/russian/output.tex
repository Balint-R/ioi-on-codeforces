Пример проверяющего модуля сначала вызывает \texttt{construct\_network(H, W, K)}.
Если \texttt{construct\_network} нарушает какие-либо ограничения, описнные в условии
задачи, то выводится одно из сообщений об ошибке, указанных в разделе Детали
реализации, и программа завершается.

Иначе пример проверяющего модуля выводит информацию двумя способами.

Во-первых, он выводит результат работы программы для робота в следующем формате:
\begin{itemize}
\item строка $1+i$ $(0 \leq i)$:  результат последней инструкции для программы робота
при запуске на изображении $i$ ($1$ или $0$).
\end{itemize}

Во-вторых, пример проверяющего модуля создаёт файл <<log.txt>> в текущем
каталоге в следующем формате:
\begin{itemize}
\item сторка $1+i$ $(0 \leq i)$: $m[i][0], m[i][1], \ldots, m[i][c-1]$
\end{itemize}

Массив в строке $1+i$ содержит значения, сохраненные в памяти робота после
запуска программы на изображении $i$. А именно, значение $m[i][j]$ содержит
значение ячейки $j$. Обратите внимание, что число $c$ (длина массива) равно $H \cdot W$
плюс число инструкций в программе для робота.


Кенан нарисовал схему зданий, расположенных вдоль главного проспекта Баку, и крытых переходов между ними. Есть $n$ зданий, которые пронумерованы от $0$ до $n-1$, и $m$ переходов, которые пронумерованы от $0$ до $m-1$. Схема нарисована на двумерной плоскости, на которой здания и переходы представлены вертикальными и горизонтальными отрезками соответственно.

Основание здания $i$ $(0 \leq i \leq n-1)$ расположено в точке $(x[i], 0)$, а само здание имеет высоту $h[i]$. Таким образом, здание является отрезком, который соединяет точки $(x[i], 0)$ и $(x[i], h[i])$.

Концы перехода $j$ $(0 \leq j \leq m-1)$ расположены в зданиях с номерами $l[j]$ и $r[j]$, а сам переход имеет положительную $y$-координату $y[j]$. Таким образом, переход является отрезком, который соединяет точки $(x[l[j]], y[j])$ и $(x[r[j]], y[j])$.

Переход и здание пересекаются, если они имеют общую точку. Таким образом, каждый переход пересекается с двумя зданиями, содержащими его концы, a также может пересекаться с другими зданиями между ними.

Кенан хочет найти длину кратчайшего пути от основания здания $s$ до основания здания $g$, предполагая, что перемещаться можно исключительно вдоль зданий и переходов, либо определить, что такого пути не существует. Обратите внимание, что запрещается перемещаться по земле, то есть по горизонтальной прямой с $y$-координатой $0$.

Разрешается перемещаться между зданием и переходом в точке их пересечения. Если концы двух переходов расположены в одной точке, разрешается переместиться из одного перехода в другой.

Помогите Кенану ответить на его вопрос.


\textbf{Детали реализации}

Вам требуется реализовать следующую функцию. Для каждого теста функция будет вызвана проверяющим модулем ровно один раз.

\begin{itemize}
\item \texttt{int64 min\_distance(int[] x, int[] h, int[] l, int[] r, int[] y,
                   int s, int g)}
\begin{itemize}
\item $x$ и $h$ : целочисленные массивы длины $n$
\item $l$, $r$ и $y$ : целочисленные массивы длины $m$
\item $s$ и $g$: два целых числа
\item Функция должна вернуть длину кратчайшего пути между основанием здания $s$ и основанием здания $g$, если такой путь существует. В противном случае функция должна вернуть $-1$.
\end{itemize}
\end{itemize}

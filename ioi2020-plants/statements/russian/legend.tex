Хазель~--- ботаник, который пришел на выставку в Сингапурский ботанический сад. На выставке находятся $n$ растений \t{различной высоты}, расположенных по кругу. Эти растения пронумерованы от $0$ до $n - 1$ в порядке обхода по часовой стрелке, после растения $n-1$ следует растение $0$.

Для каждого $i$ ($0 \leq i \leq n-1$) Хазель сравнил растение $i$ с каждым из следующих после него по часовой стрелке  $k-1$ растений и записал число $r[i]$~--- сколько из этих $k-1$ растений выше растения $i$. Таким образом, каждое значение $r[i]$ зависит от соотношения высот некоторых $k$ последовательных растений.

Например, пусть $n=5$, $k=3$ и $i=3$. Следующие $k-1 = 2$ растения в порядке обхода по часовой стрелке от растения $i = 3$~--- это растение $4$ и растение $0$. Если растение $4$ выше растения $3$, а растение $0$ ниже растения $3$, Хазель запишет $r[3] = 1$.

Будем считать, что Хазель записал числа $r[i]$ правильно. Иначе говоря, существует хотя бы одна конфигурация различных высот растений, которой соответствуют эти значения.

Вас просят сравнить высоты $q$ пар растений. К сожалению, у вас нет возможности сходить на выставку. Единственная информация, которую вы можете использовать~--- число $k$ и последовательность записанных Хазелем чисел $r[0], \ldots, r[n-1]$.

Для каждой пары различных растений $x$ и $y$, которые следует сравнить, выясните, какой из следующих трех случаев имеет место:

\begin{itemize}
\item Растение $x$ точно выше, чем растение $y$: в любой конфигурации различных высот растений $h[0], \ldots, h[n - 1]$,  для которой Хазель запишет заданный массив $r$, имеет место неравенство $h[x] > h[y]$.
\item  Растение $x$ точно ниже, чем растение $y$: в любой конфигурации различных высот растений $h[0], \ldots, h[n - 1]$, для которой Хазель запишет заданный массив $r$, имеет место неравенство $h[x] < h[y]$.
\item  Результат сравнения не определен: ни один из двух предыдущих вариантов не имеет место.
\end{itemize}

\textbf{Детали реализации}

Вам необходимо реализовать следующие функции:

\begin{itemize}
\item \t{void init(int k, int[] r)}
\begin{itemize}
\item $k$: количество последовательных растений, которые определяют значения $r[i]$.
\item $r$: массив длины $n$, где $r[i]$ задает количество растений, которые выше чем растение $i$, среди следующих после него по часовой стрелке $k-1$ растений.
\item Эта функция будет вызвана ровно один раз, до всех вызовов \t{compare\_plants}.
\end{itemize}

\item \t{int compare\_plants(int x, int y)}
\begin{itemize}
\item $x$, $y$: номера растений, которые следует сравнить.
\item Функция должна вернуть:
\begin{itemize}
  \item $1$, если растение $x$ точно выше растения $y$,
  \item $-1$, если растение $x$ точно ниже растения $y$,
  \item $0$, если результат сравнения не определен.
\end{itemize}
\item Эта функция будет вызвана ровно $q$ раз.
\end{itemize}
\end{itemize}




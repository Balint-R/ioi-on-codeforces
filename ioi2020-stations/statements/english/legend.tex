Singapore's Internet Backbone (SIB) consists of $n$ stations, which are assigned \t{indices} from $0$ to $n-1$. There are also $n-1$ bidirectional links, numbered from $0$ to $n-2$. Each link connects two distinct stations. Two stations connected with a single link are called neighbours.

A path from $x$ station to $y$ station is a sequence of distinct stations $a_0,a_1,\cdots,a_p$, such that $a_0=x$, $a_p=y$, and every two consecutive stations in the path are neighbours. There is \t{exactly one} path from any station $x$ to any other station $y$.

Any station $x$ can create a packet (a piece of data) and send it to any other station $y$, which is called the packet's \t{target}. This packet must be routed along the unique path from $x$ to $y$ as follows. Consider a station $z$ that currently holds a packet, whose target station is $y$ ($z \neq y$). In this situation
station $z$:
\begin{enumerate}
\item executes a \t{routing procedure} that determines the neighbour of $z$ which is on the unique path from $z$ to $y$, and
\item forwards the packet to this neighbour.
\end{enumerate}

However, stations have limited memory and do not store the entire list of the links in SIB to use it in
the routing procedure.

Your task is to implement a routing scheme for SIB, which consists of two procedures.

\begin{itemize}
\item The first procedure is given $n$, the list of the links in the SIB and an integer $k \geq n-1$ as the inputs. It assigns each station a \t{unique} integer \t{label} between $0$ and $k$, inclusive.
\item The second procedure is the routing procedure, which is deployed to all stations after labels are assigned. It is given \t{only} the following inputs:
\begin{itemize}
\item $s$, the \t{label} of the station that currently holds a packet,
\item $t$, the \t{label} of the packet's target station ($t \neq s$),
\item $c$, the list of the \t{labels} of all neighbours of $s$.
\end{itemize}
It should return the \t{label} of the neighbour of $s$ that the packet should be forwarded to.

\end{itemize}

In one subtask, the score of your solution depends on the value of the maximum label assigned to any station (in general, smaller is better).

\textbf{Implementation details}

You should implement the following procedures:

\begin{itemize}
\item \t{int[] label(int n, int k, int[] u, int[] v)}
\begin{itemize}
\item $n$: number of stations in the SIB.
\item $k$: maximum label that can be used.
\item $u$ and $v$:  arrays of size $n-1$, describing the links. For each $i$ ($0 \leq i \leq n-2$), link $i$ connects stations with indices $u[i]$ and $v[i]$.
\item This procedure should return a single array $L$ of size $n$. For each $i$ ($0 \leq i \leq n-1$) $L[i]$ is the label assigned to station with index $i$. All elements of array $L$ must be unique and between $0$ and $k$, inclusive.
\end{itemize}
\item \t{int find\_next\_station(int s, int t, int[] c)}
\begin{itemize}

\item $s$: label of the station holding a packet.
\item $t$: label of the packet's target station.
\item $c$: an array giving the list of the labels of all neighbours of $s$. The array $c$ is sorted in ascending order.
\item This procedure should return the label of a neighbour of $s$ that the packet should be forwarded to.
\end{itemize}
\end{itemize}

Each test case involves one or more independent scenarios (i.e., different SIB descriptions). For a test case involving $r$ scenarios, a \t{program} that calls the above procedures is run exactly two times,
as follows.

During the first run of the program:
\begin{itemize}
\item \t{label} procedure is called $r$ times,
\item the returned labels are stored by the grading system, and
\item \t{find\_next\_station} is not called.
\end{itemize}

During the second run of the program:
\begin{itemize}
\item \t{find\_next\_station} may be called multiple times. In each call, an \t{arbitrary} scenario is chosen, and the labels returned by the call to \t{label} procedure in that scenario are used as the inputs to \t{find\_next\_station}.
\item \t{label} is not called.
\end{itemize}

In particular, any information saved to static or global variables in the first run of the program is not available within \t{find\_next\_station} procedure.


Проверяющий модуль считывает данные в следующем формате:
\begin{itemize}
\item строка $1$: $r$ ($1 \leq r \leq 10$)
\end{itemize}

Далее следуют $r$ блоков данных, каждый описывает один из сценариев. Каждый из блоков имеет следующий формат:
\begin{itemize}
\item строка $1$: $n\ k$ ($2 \leq n \leq 1000$, $k \geq n-1$)
\item строка $2+i$ ($0 \leq i \leq n - 2$): $u[i]\ v[i]$ ($0 \leq u[i], v[i] \leq n - 1$)
\item строка $1 + n$: $q$: количество вызовов функции \t{find\_next\_station}.
\item строка $2 + n + j$ ($0 \leq j \leq q - 1$): $z[j]\ y[j]\ w[j]$: \t{индексы} станций, используемых в $j$-м вызове функции  \t{find\_next\_station}. Станция $z[j]$ содержит пакет в текущий момент, станция $y[j]$ является пунктом назначения, а станция $w[j]$ обозначает место, куда пакет должен быть передан дальше.
\end{itemize}

Для каждого вызова функции \t{find\_next\_station}, входные данные соответствуют одному произвольно выбранному вызову функции \t{label}. Рассмотрим идентификаторы, которые получились в результате этого вызова. Тогда:
\begin{itemize}
\item $s$ и $t$ обозначают идентификаторы двух различных станцией.
\item $c$ содержит идентификаторы всех соседей станции с идентификатором $s$, отсортированные по возрастания.
\end{itemize}

Для каждого из тестовых примеров общая длина всех массивов $c$, переданных функции \t{find\_next\_station}, не превышает $100\,000$, суммарно для всех сценариев.
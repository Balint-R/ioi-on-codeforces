\textbf{Пример 1}

Рассмотрим следующий вызов функции:

\t{find\_maximum(2, [[0, 2, 5],[1, 1, 3]])}

Данный вызов означает, что:
\begin{itemize}
\item в розыгрыше будет $k=2$ раунда;
\item на карточках цвета $0$ написаны числа $0$, $2$ и $5$;
\item на карточках цвета $1$ написаны числа $1$, $1$ и $3$.
\end{itemize}

Один из возможных вариантов выбора билетов, который максимизирует суммарный размер призов, такой:
\begin{itemize}
\item В раунде $0$ Ринго выбирает билет $0$ цвета $0$ (с числом $0$) и билет $2$ цвета $1$ (с числом $3$). Минимальный возможный размер приза за этот раунд равен $3$. Например, ведущий может выбрать $b = 1$: $|1 - 0| + |1 - 3| = 1 + 2 = 3$.
\item В раунде $1$ Ринго выбирает билет $2$ цвета $0$ (с числом $5$) и билет $1$ цвета $1$ (с числом $1$). Минимальный возможный размер приза за этот раунд равен $4$. Например, ведущий может выбрать $b = 3$: $|3 - 1| + |3 - 5| = 2 + 2 = 4$.
\item Таким образом, суммарный размер призов равен $3 + 4 = 7$.
\end{itemize}


Чтобы выбрать такие билеты, функция \t{find\_maximum} должна вызвать \t{allocate\_tickets} следующим образом:
\begin{itemize}
\item \t{allocate\_tickets([[0, -1, 1], [-1, 1, 0]])}
\end{itemize}

Функция \t{find\_maximum} должна вернуть $7$.

\textbf{Пример 2}

Рассмотрим следующий вызов функции:

\t{find\_maximum(1, [[5, 9], [1, 4], [3, 6], [2, 7]])}

Данный вызов означает, что:
\begin{itemize}
\item в розыгрыше только один раунд,
\item на карточках цвета $0$ написаны числа $5$ и $9$;
\item на карточках цвета $1$ написаны числа $1$ и $4$;
\item на карточках цвета $2$ написаны числа $3$ и $6$;
\item на карточках цвета $3$ написаны числа $2$ и $7$.
\end{itemize}

Один из возможных вариантов выбора билетов, который максимизирует суммарный размер призов, такой:
\begin{itemize}
\item В раунде $0$ Ринго выбирает билет $1$ цвета $0$ (с числом $9$), билет $0$ цвета $1$ (с числом $1$), билет $0$ цвета $2$ (с числом $3$), и билет $1$ цвета $3$ (с числом $7$). Минимальный возможный размер приза за этот раунд равен $12$, если ведущий выберет число $b = 3$: $|3 - 9| + |3 - 1| + |3 - 3| + |3 - 7| = 6 + 2 + 0 + 4 = 12$.
\end{itemize}

Чтобы выбрать такие билеты, функция \t{find\_maximum} должна вызвать \t{allocate\_tickets} следующим образом:
\begin{itemize}
\item \t{allocate\_tickets([[-1, 0], [0, -1], [0, -1], [-1, 0]])}
\end{itemize}

Функция \t{find\_maximum} должна вернуть $12$.
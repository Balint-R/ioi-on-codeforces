Грейс~--- биолог, она работает в компании, занимающейся биоинформатикой в Сингапуре. На работе она занимается анализом ДНК различных организмов. В этой задаче последовательность ДНК определена как строка, состоящая из символов <<A>>, <<T>> и <<C>>. Обратите внимание, что в этой задаче последовательности ДНК \textbf{не могут содержать символ <<G>>}.

Определим мутацию как операцию на последовательности ДНК, в результате которой два произвольных элемента этой последовательности меняются местами. Например, в результате мутации можно преобразовать последовательность <<A\textbf{C}T\textbf{A}>> в последовательность <<A\textbf{A}T\textbf{C}>>, поменяв местами выделенные жирным символы <<A>> и <<C>>.

Мутационным расстоянием между двумя последовательностями назовем минимальное число мутаций, необходимое для превращения одной последовательности ДНК в другую, либо число $-1$, если превратить одну последовательность ДНК в другую с помощью мутаций нельзя.

Грейс анализирует две последовательности ДНК $a$ и $b$, каждая из которых состоит из $n$ элементов, проиндексированных от $0$ до $n - 1$.
Ваша задача~--- помочь Грейс ответить на $q$ запросов следующего формата: чему равно мутационное расстояние между подстрокой $a[x\ldots y]$ и подстрокой $b[x\ldots y]$?
Здесь подстрока $s[x\ldots y]$ для последовательности ДНК $s$ определяется как последовательность подряд идущих символов $s$ с индексами от $x$ до $y$, включительно.
Другими словами, $s[x\ldots y]$ представляет собой последовательность $s[x]s[x+1]\ldots s[y]$.

\textbf{Детали реализации}

Вам следует реализовать следующие функции:

\begin{itemize}
\item \texttt{void init(string a, string b)}
\begin{itemize}

\item $a$, $b$: строки длины $n$, задающие две последовательности ДНК, которые необходимо проанализировать.
\item Эта функция будет вызвана ровно один раз, до вызовов функции \texttt{get\_distance}.
\end{itemize}

\item \texttt{int get\_distance(int x, int y)}
\begin{itemize}
\item $x$, $y$: начальный и конечный индекс подстрок, которые необходимо проанализировать.
\item Функция должна вернуть мутационное расстояние между $a[x..y]$ и $b[x..y]$.
\item Эта функция будет вызвана $q$ раз.
\end{itemize}
\end{itemize}






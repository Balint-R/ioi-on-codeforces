\textbf{Пример}

Рассмотрим следующий вызов функции:

\texttt{init("ATACAT", "ACTATA")}

Пусть затем грейдер делает вызов функции \texttt{get\_distance(1, 3)}.
Этот вызов должен вернуть мутационное расстояние между $a[1\ldots 3]$ и $b[1\ldots 3]$, то есть между последовательностями ДНК <<TAC>> и <<CTA>>.
<<TAC>> может быть преобразована в <<CTA>> с помощью $2$ мутаций: <<\textbf{T}A\textbf{C}>> $\rightarrow$ <<\textbf{C}A\textbf{T}>>, затем <<C\textbf{AT}>> $\rightarrow$ <<C\textbf{TA}>>, а выполнить преобразование менее, чем за $2$ мутации, невозможно.

Следовательно, данный вызов должен вернуть $2$.

Пусть затем грейдер делает вызов функции \texttt{get\_distance(4, 5)}.
Этот вызов должен вернуть мутационное расстояние между <<AT>> и <<TA>>.
<<AT>> может быть преобразована в <<TA>> за одну мутацию, и ясно, что требуется хотя бы одна мутация.

Следовательно, данный вызов должен вернуть $1$.

Наконец, пусть затем грейдер делает вызов функции \texttt{get\_distance(3, 5)}.
Поскольку \textbf{не существует} способа преобразовать последовательность <<CAT>> в <<ATA>> с помощью последовательности мутаций, этот вызов должен вернуть $-1$.


В парке неподалеку находятся $n$ \textbf{фонтанов}, пронумерованных от $0$ до $n-1$.
Будем считать, что фонтаны представляют собой точки на плоскости.
А именно, фонтан с номером $i$ ($0\leq i\leq n-1$) находится в точке $(x[i],y[i])$, где $x[i]$ и $y[i]$~--- \textbf{четные целые числа}.
Все фонтаны находятся в различных точках.

Архитектора Тимати наняли для того, чтобы проложить несколько \textbf{дорожек}, а также разместить по одной \textbf{лавочке} на каждой из дорожек.
Каждая дорожка должна представлять собой \textbf{горизонтальный} или \textbf{вертикальный} отрезок длины $2$, в концах которого находятся два различных фонтана. Дорожки должны быть проложены таким образом, чтобы от любого фонтана до любого другого можно было дойти по дорожкам. Исходно в парке нет дорожек.

На каждой дорожке необходимо разместить \textbf{ровно одну} лавочку, которая будет \textbf{размещаться на этой дорожке}.
Каждая лавочка должна быть расположена в точке $(a, b)$, где $a$ и $b$~--- \textbf{нечетные целые числа}.
Все точки, в которых будут расположены лавочки, должны быть \textbf{различными}.
Лавочка в точке $(a,b)$ может размещаться на некоторой дорожке, если \textbf{оба конца} этой дорожки находятся в множестве $(a-1,b-1)$, $(a-1,b+1)$, $(a+1,b-1)$ и $(a+1,b+1)$.
Например, лавочка в точке $(3,3)$ может размещаться на дорожке, если эта дорожка представляет собой один из следующих отрезков: $(2,2)$~--- $(2,4)$, $(2,4)$~--- $(4,4)$, $(4,4)$~--- $(4,2)$, $(4,2)$~--- $(2,2)$.

Помогите Тимати выяснить, можно ли проложить дорожки и разместить на них лавочки в соответствии с описанными выше условиями. Если это возможно, необходимо вернуть пример подходящего решения. Если есть несколько возможных решений, можно вернуть любое из них.

\textbf{Детали реализации}

Вам необходимо реализовать следующую функцию:

\begin{itemize}
\item \texttt{int construct\_roads(int[] x, int[] y)}
\begin{itemize}
\item $x,y$: два массива длины $n$. Для каждого $i$ ($0\leq i\leq n-1$) фонтан $i$ находится в точке $(x[i],y[i])$, где $x[i]$ и $y[i]$~--- четные целые числа.
\item Если решение существует, то эта функция должна сделать ровно один вызов функции \texttt{build} (описанной ниже), чтобы описать решение, а затем функция должна вернуть $1$.
\item В противном случае функция должна вернуть $0$, не вызывая функцию \texttt{build}.
\item Эта функция будет вызвана ровно один раз.
\end{itemize}
\end{itemize}

Ваша реализация должна вызывать следующую функцию, чтобы описать решение~--- проложенные дорожки и размещенные на них лавочки:

\begin{itemize}

\item \texttt{void build(int[] u, int[] v, int[] a, int[] b)}
\begin{itemize}

\item Пусть $m$ обозначает общее проложенных дорожек.
\item $u,v$: два массива длины $m$, описывающие дорожки, которые необходимо проложить. Эти дорожки пронумерованы от $0$ до $m-1$. Для каждого $j$ ($0 \leq j \leq m-1$) дорожка $j$ соединяет фонтаны $u[j]$ и $v[j]$. Каждая дорожка должна представлять собой горизонтальный или вертикальный отрезок длины $2$. Любые две различные дорожки могут иметь только одну общую точку~--- один из концов (фонтан). Дорожки должны быть проложены таким образом, чтобы от любого фонтана до любого другого можно было добраться по дорожкам.
\item $a,b$: два массива длины $m$, описывающие лавочки. Для каждого $j$ ($0 \leq j \leq m-1$) лавочка, которая размещается на дорожке $j$, расположена в точке $(a[j],b[j])$. Никакие две различные лавочки не должны размещаться в одной точке.
\end{itemize}
\end{itemize}
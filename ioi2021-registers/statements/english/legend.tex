Christopher the engineer is working on a new type of computer processor.

The processor has access to $m$ different $b$-bit memory cells (where $m = 100$ and $b=2000$),
which are \textbf{called} registers, and are numbered from $0$ to $m-1$. We denote the registers by $r[0], r[1], \ldots, r[m-1]$.
Each register is an array of $b$ bits, numbered from $0$ (the rightmost bit) to
$b-1$ (the leftmost bit). For each $i$ $(0\leq i \leq m-1)$ and each $j$ $(0 \leq j \leq b-1)$ we denote the $j$-th bit of register $i$ by $r[i][j]$.

For any sequence of bits $d\_0, d\_1, \ldots, d\_{l-1}$ (of arbitrary length $l$) the \textbf{integer value} of the sequence
is equal to $2^0 \cdot d\_0 + 2^1 \cdot d\_1 + \ldots + 2^{l-1} \cdot d\_{l-1}$.
We say that the \textbf{integer value stored in a register}
$i$ is the integer value of the sequence of its bits, i.e., it is $2^0 \cdot r[i][0] + 2^1 \cdot r[i][1] + \ldots + 2^{b-1} \cdot r[i][b-1]$.

The processor has $9$ types of \textbf{instructions} that can be used to modify the bits in the registers. Each
instruction operates on one or more registers and stores the output in one of the registers. In the
following, we use $x := y$ to denote an operation of changing the value of $x$ such that it becomes
equal to $y$. The operations performed by each type of instruction are described below
\begin{itemize}
\item $move(t,y)$: copy the array of bits in register $y$ to register $t$. For each $j$ $(0 \leq j \leq b-1)$ set $r[t][j] := r[y][j]$.

\item $store(t,v)$: set register $t$ to be equal to $v$, where $v$ is an array of $b$ bits. For each $j$ $(0 \leq j \leq b-1)$ set $r[t][j] := v[j]$.

\item $and(t,x,y)$: take the bitwise-AND of registers $x$ and $y$, and store the result in register $t$. For
each  $j$ $(0 \leq j \leq b-1)$ set $r[t][j] := 1$ if \textbf{both} $r[x][j]$ and $r[y][j]$ are $1$, and set $r[t][j] := 0$ otherwise.

\item $or(t,x,y)$: Take the bitwise-OR of registers $x$ and $y$, and store the result in register $t$. For each $j$ $(0 \leq j \leq b-1)$ set $r[t][j] := 1$ if \textbf{at least one} of $r[x][j]$ and $r[y][j]$ are $1$, and
set $r[t][j] := 0$ otherwise.

\item $xor(t,x,y)$: Take the bitwise-XOR of registers $x$ and $y$, and store the result in register $t$. For
each  $j$ $(0 \leq j \leq b-1)$ set $r[t][j] := 1$ if \textbf{exactly one} of $r[x][j]$ and $r[y][j]$ is $1$, and set $r[t][j] := 0$ otherwise.

\item $not(t,x)$: Take the bitwise-NOT of register $x$, and store the result in register $t$. For each $j$ $(0 \leq j \leq b-1)$ set $r[t][j] := 1-r[x][j]$.

\item $left(t,x,p)$: Shift all bits in register $x$ to the left by $p$, and store the result in register $t$. The
result of shifting the bits in register $x$ to the left by $p$ is an array $v$ consisting of $b$ bits. For each $j$ $(0 \leq j \leq b-1)$, $v[j] = r[x][j-p]$, if $j \geq p$, and $v[j] = 0$ otherwise. For all $j$ $(0 \leq j \leq b-1)$ set $r[t][j] := v[j]$.

\item $right(t,x,p)$: Shift all bits in register $x$ to the right by $p$, and store the result in register $t$. The result of shifting the bits in register $x$ to the right by $p$ is an array $v$ consisting of $b$ bits. For
each  $j$ $(0 \leq j \leq b-1)$ $v[j] = r[x][j+p]$, if $j \leq b - 1 - p$, and $v[j] = 0$ otherwise. For all $j$ $(0 \leq j \leq b-1)$ set $r[t][j] := v[j]$.

\item $add(t,x,y)$: Add the integer values stored in register $x$ and register $y$, and store the result in register $t$. The addition is carried out modulo $2^b$.
Formally, let $X$ be the integer value stored in
register $x$, and $Y$ be the integer value stored in register $y$ before the operation. Let $T$ be the
integer value stored in register $t$ after the operation. If  $X+Y < 2^b$,  set the bits of $t$, such that $T = X+Y$.
Otherwise, set the bits of $t$, such that $T=X+Y-2^b$.
\end{itemize}


Christopher would like you to solve two types of tasks using the new processor. The type of a task is
denoted by an integer $s$. For both types of tasks, you need to produce a \textbf{program}, that is a sequence
of instructions defined above.

The \textbf{input} to the program consists of $n$ integers $a[0],a[1],\ldots,a[n-1]$each having $k$ bits, i.e., $a[i] < 2^k$ ($0 \leq i \leq n-1$).
Before the program is executed, all of the input numbers are stored
sequentially in register $0$, such that for each $i$ $(0 \leq i \leq n-1)$ the integer value of the sequence of $k$ bits $r[0][i \cdot k], r[0][i \cdot k + 1], \ldots, r[0][(i+1) \cdot k - 1]$ is equal to $a[i]$. Note that $n \cdot k \leq b$.
All other bits in register $0$ (i.e., those with indices between $n \cdot k$ and$b-1$, inclusive) and all bits in all
other registers are initialized to $0$.

Running a program consists in executing its instructions in order. After the last instruction is executed,
the \textbf{output} of the program is computed based on the final value of bits in register $0$. Specifically, the
output is a sequence of $n$ integers $c[0], c[1], \ldots, c[n-1]$, where for each $i$ ($0 \leq i \leq n-1$) $c[i]$ is the integer value of a sequence consisting of bits $i\cdot k$ to $(i + 1)\cdot k-1$ of register $0$.
Note that
after running the program the remaining bits of register $0$ (with indices at least $n \cdot k$) and all bits of all
other registers can be arbitrary.

\begin{itemize}
\item The first task $(s=0)$ s to find the smallest integer among the input integers $a[0],a[1],\ldots,a[n-1]$. Specifically, $c[0]$ must be the minimum of $a[0], a[1], \ldots, a[n-1]$. The values of $c[1], c[2], \ldots, c[n-1]$ can be arbitrary.

\item The second task $(s=1)$ is to sort the input integers $a[0],a[1],\ldots,a[n-1]$ in
nondecreasing order. Specifically, for each $i$ ($0\leq i\leq n-1$), $c[i]$ should be equal to the $(1+i)$-th smallest integer among $a[0],a[1],\ldots, a[n-1]$ (i.e., $c[0]$ is the smallest integer among the input integers).
\end{itemize}


Provide Christopher with programs, consisting of at most $q$ instructions each, that can solve these
tasks.

\textbf{Implementation Details}

You should implement the following procedure:
\begin{itemize}
\item \texttt{void construct\_instructions(int s, int n, int k, int q)}
\begin{itemize}
\item $s$: type of task.
\item $n$: number of integers in the input
\item $k$: number of bits in each input integer.
\item $q$: maximum number of instructions allowed.
\item This procedure is called exactly once and should construct a sequence of instructions to perform
the required task.
\end{itemize}
\end{itemize}

This procedure should call one or more of the following procedures to construct a sequence of
instructions:

\begin{itemize}
\item \texttt{void append\_move(int t, int y)}
\item \texttt{void append\_store(int t, bool[] v)}
\item \texttt{void append\_and(int t, int x, int y)}
\item \texttt{void append\_or(int t, int x, int y)}
\item \texttt{void append\_xor(int t, int x, int y)}
\item \texttt{void append\_not(int t, int x)}
\item \texttt{void append\_left(int t, int x, int p)}
\item \texttt{void append\_right(int t, int x, int p)}
\item \texttt{void append\_add(int t, int x, int y)}
\begin{itemize}
\item Each procedure appends a $move(t,y)$ $store(t,v)$, $and(t,x,y)$, $or(t,x,y)$, $xor(t,x,y)$, $not(t,x)$, $left(t,x,p)$, $right(t,x,p)$ or $add(t,x,y)$ instruction to the program,
respectively.
\item For all relevant instructions, $t$, $x$, $y$ must be at least $0$ and at most $m-1$.
\item For all relevant instructions, $t$, $x$, $y$ are not necessarily pairwise distinct.
\item For $left$ and $right$ instructions, $p$ must be at least $0$ and at most $b$.
\item For $store$ instructions, the length of $v$ must be $b$.
\end{itemize}
\end{itemize}

You may also call the following procedure to help you in testing your solution:

\begin{itemize}
\item \texttt{void append\_print(int t)}
\begin{itemize}
\item Any call to this procedure will be ignored during the grading of your solution.
\item In the sample grader, this procedure appends a $print(t)$ operation to the program
\item When the sample grader encounters a $print(t)$ operation during the execution of a program, it
prints $n$ $k$-bit integers formed by the first $n \cdot k$ bits of register $t$ (see ``Sample Grader'' section
for details).
\item $t$ must satisfy $0 \leq t \leq m-1$.
\item Any call to this procedure does not add to the number of constructed instructions.
\end{itemize}
\end{itemize}

After appending the last instruction, \texttt{construct\_instructions} should return. The program is then
evaluated on some number of test cases, each specifying an input consisting of $n$ $k$-bit integers $a[0], a[1], \ldots, a[n-1]$.
Your solution passes a given test case if the output of the program $c[0], c[1], \ldots, c[n-1]$  for the provided input satisfies the following conditions:
\begin{itemize}
\item if $s = 0$, $c[0]$ should be the smallest value among $a[0], a[1], \ldots, a[n-1]$.
\item if $s = 1$, for all $i$ ($0\leq i\leq n-1$) $c[i]$ should be the $(1+i)$-th smallest integer among $a[0],a[1],\ldots, a[n-1]$.
\end{itemize}


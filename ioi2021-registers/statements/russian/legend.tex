Инженер Кристофер работает над новым типом процессора.

Процессор имеет доступ к $m$ различным $b$-битным ячейкам памяти (где $m = 100$ и $b=2000$), которые называются \textbf{регистрами} и пронумерованы числами от $0$ до $m-1$. Мы будем обозначать регистры как $r[0], r[1], \ldots, r[m-1]$.
Каждый регистр представляет собой массив из $b$ битов, пронумерованных от $0$ (самый правый бит) до $b-1$ (самый левый бит).
Для каждого $i$ $(0\leq i \leq m-1)$ и каждого $j$ $(0 \leq j \leq b-1)$ обозначим $j$-й бит регистра $i$ как $r[i][j]$.

Для любой последовательности битов $d\_0, d\_1, \ldots, d\_{l-1}$ (произвольной длины $l$) определим \textbf{целочисленное значение} этой последовательности как $2^0 \cdot d\_0 + 2^1 \cdot d\_1 + \ldots + 2^{l-1} \cdot d\_{l-1}$.
Определим также \textbf{целочисленное значение регистра} $i$ как целочисленное значение последовательности бит этого регистра, то есть значение $2^0 \cdot r[i][0] + 2^1 \cdot r[i][1] + \ldots + 2^{b-1} \cdot r[i][b-1]$.

Процессор имеет $9$ типов \textbf{инструкций}, которые могут быть использованы для изменения значений регистров. Каждая инструкция использует один или несколько регистров и записывает результат в один из регистров. В дальнейшем мы будем использовать обозначение $x := y$ для операции присваивания биту $x$ значения $y$. Операции, выполняемые каждой из инструкций, представлены ниже:
\begin{itemize}
\item $move(t,y)$: скопировать биты регистра $y$ в регистр $t$. Для каждого $j$ $(0 \leq j \leq b-1)$ выполняется $r[t][j] := r[y][j]$.

\item $store(t,v)$: присвоить биты регистра $t$ значению $v$, где $v$~--- некоторый массив из $b$ бит. Для каждого $j$ $(0 \leq j \leq b-1)$ выполняется присваивание $r[t][j] := v[j]$.

\item $and(t,x,y)$: вычислить побитовое И регистров $x$ и $y$ и записать результат в регистр $t$. Для каждого $j$ $(0 \leq j \leq b-1)$ выполняется присваивание $r[t][j] := 1$ если \textbf{оба} значения $r[x][j]$ и $r[y][j]$ равны $1$, и $r[t][j] := 0$ в противном случае.

\item $or(t,x,y)$: вычислить побитовое ИЛИ регистров $x$ и $y$ и записать результат в регистр $t$. Для каждого $j$ $(0 \leq j \leq b-1)$ выполняется присваивание $r[t][j] := 1$ если \textbf{хотя бы одно} из значений $r[x][j]$ и $r[y][j]$ равны $1$, и $r[t][j] := 0$ в противном случае.

\item $xor(t,x,y)$: вычислить побитовое исключающее ИЛИ регистров $x$ и $y$ и записать результат в регистр $t$. Для каждого $j$ $(0 \leq j \leq b-1)$ выполняется присваивание $r[t][j] := 1$ если \textbf{ровно одно} из значений $r[x][j]$ и $r[y][j]$ равно $1$, и $r[t][j] := 0$ в противном случае.

\item $not(t,x)$: вычислить побитовое отрицание регистра $x$ и записать результат в регистр в $t$. Для каждого $j$ $(0 \leq j \leq b-1)$ выполняется присваивание $r[t][j] := 1-r[x][j]$.

\item $left(t,x,p)$: сдвинуть все биты из регистра $x$ влево на $p$ и записать результат в регистр $t$. Результат сдвига битов из регистра $x$ на $p$ влево представляет собой массив $v$ из $b$ бит. Для каждого $j$ $(0 \leq j \leq b-1)$ значение $v[j] = r[x][j-p]$, если $j \geq p$, и $v[j] = 0$ в противном случае. Для каждого $j$ $(0 \leq j \leq b-1)$ выполняется присваивание $r[t][j] := v[j]$.

\item $right(t,x,p)$: сдвинуть все биты из регистра $x$ вправо на $p$ и записать результат в регистр $t$. Результат сдвига битов из регистра $x$ вправо на $p$ представляет собой массив $v$ из $b$ битов. Для каждого $j$ $(0 \leq j \leq b-1)$ значение $v[j] = r[x][j+p]$, если $j \leq b - 1 - p$, и $v[j] = 0$ в противном случае. Для каждого $j$ $(0 \leq j \leq b-1)$ выполняется присваивание $r[t][j] := v[j]$.

\item $add(t,x,y)$: вычислить сумму целочисленных значений регистров $x$ и $y$ и записать результат в регистр $t$. Сложение выполняется по модулю $2^b$.
Более формально, пусть $X$ обозначает целочисленное значение регистра $x$, а $Y$~--- целочисленное значение регистра $y$ перед выполнением операции. Пусть $T$ обозначает целочисленное значение регистра $t$ после выполнения операции. Если $X+Y < 2^b$, то биты регистра $t$ выставляются таким образом, чтобы $T = X+Y$.
В противном случае биты регистра $t$ выставляются таким образом, что $T=X+Y-2^b$.
\end{itemize}


Кристофер хотел бы научиться решать два типа задач с помощью нового процессора. Тип задачи обозначается числом $s$. Для обоих типов задач вам необходимо предоставить \textbf{программу}, которая представляет собой последовательность операций, описанных выше.

\textbf{Входные данные} к программе представляют собой $n$ чисел $a[0],a[1],\ldots,a[n-1]$, каждое из которых состоит из $k$-бит, то есть, $a[i] < 2^k$ ($0 \leq i \leq n-1$).
Перед выполнением программы все числа из входных данных содержаться последовательно в регистре $0$, а именно, для каждого $i$ $(0 \leq i \leq n-1)$ целочисленное значение последовательности из $k$ бит $r[0][i \cdot k], r[0][i \cdot k + 1], \ldots, r[0][(i+1) \cdot k - 1]$ равно $a[i]$. Обратите внимание, что $n \cdot k \leq b$.
Все остальные биты в регистре $0$ (биты с индексами между $n \cdot k$ и $b-1$, включительно) и все биты в других регистрах изначально равны $0$.

Выполнение программы состоит из выполнения инструкций этой программы по порядку.
После выполнения последней инструкции \textbf{выходные данные} определяются итоговыми значениями битов регистра $0$.
Более конкретно, результатом работы программы является массив из $n$ чисел $c[0], c[1], \ldots, c[n-1]$, где для каждого $i$ ($0 \leq i \leq n-1$) значение $c[i]$ это целочисленное значение последовательности бит от $i\cdot k$ до $(i + 1)\cdot k-1$ регистра $0$.
Обратите внимание, что после выполнения программы все остальные биты регистра $0$ (с индексами хотя бы $n \cdot k$) и все остальные биты других регистров могут иметь произвольные значения.

\begin{itemize}
\item Первая задача $(s=0)$ состоит в нахождении минимального числа среди $a[0],a[1],\ldots,a[n-1]$. Более конкретно, $c[0]$ должно быть равно минимум среди $a[0], a[1], \ldots, a[n-1]$. Значения $c[1], c[2], \ldots, c[n-1]$ могут быть произвольными.

\item Вторая задача $(s=1)$ состоит в сортировке чисел $a[0],a[1],\ldots,a[n-1]$ по неубыванию Более конкретно, для каждого $i$ ($0\leq i\leq n-1$), $c[i]$ должно быть равно $(1+i)$-му числу в порядке сортировки $a[0],a[1],\ldots, a[n-1]$ (то есть, $c[0]$ должно быть равно наименьшему из чисел).
\end{itemize}


Помогите Кристоферу написать программы для решения обеих задач, состоящие из не более, чем $q$ инструкций.

\textbf{Детали реализации}

Вам необходимо реализовать функцию:
\begin{itemize}
\item \texttt{void construct\_instructions(int s, int n, int k, int q)}
\begin{itemize}
\item $s$: тип задачи.
\item $n$: количество чисел во входных данных.
\item $k$: количество бит в каждом из чисел из входных данных.
\item $q$: максимальное доступное число инструкций.
\item Функция будет вызвана ровно один раз и должна построить последовательность инструкций для выполнения нужной задачи. 
\end{itemize}
\end{itemize}

Для построения последовательности инструкций функция может вызывать любую из перечисленных ниже функций:

\begin{itemize}
\item \texttt{void append\_move(int t, int y)}
\item \texttt{void append\_store(int t, bool[] v)}
\item \texttt{void append\_and(int t, int x, int y)}
\item \texttt{void append\_or(int t, int x, int y)}
\item \texttt{void append\_xor(int t, int x, int y)}
\item \texttt{void append\_not(int t, int x)}
\item \texttt{void append\_left(int t, int x, int p)}
\item \texttt{void append\_right(int t, int x, int p)}
\item \texttt{void append\_add(int t, int x, int y)}
\begin{itemize}
\item Каждый из вызовов добавляет в список инструкций $move(t,y)$ $store(t,v)$, $and(t,x,y)$, $or(t,x,y)$, $xor(t,x,y)$, $not(t,x)$, $left(t,x,p)$, $right(t,x,p)$ или $add(t,x,y)$, соответственно.
\item Для всех инструкций $t$, $x$, $y$ должны быть хотя бы $0$ и не больше $m-1$.
\item Для всех инструкций $t$, $x$, $y$ не обязательно должны быть попарно различными.
\item Для инструкций $left$ и $right$ значение $p$ должно быть хотя бы $0$ и не более $b$.
\item Для инструкций $store$ длина массива $v$ должна быть равна $b$.
\end{itemize}
\end{itemize}

Вы можете также использовать следующую функцию для облегчения тестирования своего решения:

\begin{itemize}
\item \texttt{void append\_print(int t)}
\begin{itemize}
\item Любой вызов этой функции будет проигнорирован при проверке вашей программы проверяющей системой.
\item В доступном вам грейдере эта функция добавляет инструкцию $print(t)$ в программу.
\item Когда грейдер встречает инструкцию $print(t)$, он печатает $n$ $k$-битных чисел, которые образованы первыми $n \cdot k$ битами регистра $t$ (более подробное описание находится в секции <<Пример грейдера>>).
\item $t$ должно удовлетворять неравенствам $0 \leq t \leq m-1$.
\item Любой вызов этой функции не учитывается при вычислении общего числа инструкций.
\end{itemize}
\end{itemize}

Функция \texttt{construct\_instructions} должна завершить работу после добавления последней инструкции. После этого программа будет запущена на некотором количестве тестовых примеров, каждый из которых состоит из $n$ $k$-битных чисел $a[0], a[1], \ldots, a[n-1]$.
Решение считается прошедшим тест, если результирующий массив $c[0], c[1], \ldots, c[n-1]$ удовлетворяет следующим условиям:
\begin{itemize}
\item если $s = 0$, то $c[0]$ должно быть равно наименьшему из чисел $a[0], a[1], \ldots, a[n-1]$.
\item если $s = 1$, то для всех $i$ ($0\leq i\leq n-1$) $c[i]$ должно быть равно $(1+i)$-му числу в порядке сортировки $a[0],a[1],\ldots, a[n-1]$.
\end{itemize}

